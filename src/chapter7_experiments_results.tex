\chapter{Experiments and Results}
\label{chap:experiments_results}

This chapter details the experimental methodologies and presents the findings from a series of evaluations designed to assess various components and strategies within Retrieval-Augmented Generation systems. The experiments cover a range of techniques from fundamental retrieval and chunking methods to advanced reranking, prompt engineering, and the impact of different embedding and generative models. The overarching goal is to identify optimal configurations for robust and effective RAG pipelines.

\section{General Experimental Setup}
\label{sec:general_setup}
To ensure a systematic and reproducible evaluation, all experiments were conducted using a consistent setup.

\subsection{Dataset(s)}
The primary dataset used for these experiments is a curated collection of academic papers and articles related to the field of artificial intelligence. This dataset was chosen for its complexity, technical vocabulary, and the need for precise, fact-based answers. For question generation and evaluation, a set of 100 questions was manually crafted, covering a range of topics within the dataset. Each question is designed to have a verifiable answer within the document corpus.

\subsection{Baseline Configuration}
To measure the impact of different optimization techniques, a baseline RAG configuration was established:
\begin{itemize}
    \item \textbf{Embedding Model:} OpenAI \texttt{text-embedding-ada-002}, a widely used and strong baseline model.
    \item \textbf{Chunking Strategy:} Naive fixed-size chunking with a chunk size of 300 tokens and an overlap of 50 tokens.
    \item \textbf{Retrieval Strategy:} Basic cosine similarity search with a fixed retrieval of the top 5 most similar chunks (Top-N).
    \item \textbf{Generator LLM:} \texttt{GPT-4}.
    \item \textbf{Prompt Template:} A standard zero-shot prompt instructing the LLM to answer the question based on the provided context.
\end{itemize}

\subsection{Core Evaluation Metrics}
The performance of the RAG system was evaluated at both the retrieval and generation stages.
\begin{itemize}
\item \textbf{Retrieval Performance:} Assessed using \textit{Precision@k}, \textit{Recall@k}, and \textit{Mean Reciprocal Rank (MRR)}. These metrics are crucial for understanding the effectiveness of the retrieval stage in isolation.
\item \textbf{Generation Quality:} Evaluated using the \textit{LLM-as-a-Judge} method with the DeepEval framework. The key metrics tracked were \textit{Faithfulness}, \textit{Answer Relevancy}, and \textit{Hallucination Rate} \autocite{zheng2023judging}.
\end{itemize}

\section{Experiment 1: Embedding Models}
\label{sec:exp_embedding_models}
This experiment evaluates the performance of different embedding models on long-context retrieval tasks. The goal is to understand how different architectural and chunking strategies affect retrieval performance on documents of varying lengths and complexity.

\subsection{Models and Methodology}
We compared five embedding models on the \textbf{LongEmbed} benchmark \autocite{zhu2024longembedextendingembeddingmodels}, a suite of datasets designed to test long-context retrieval. The models evaluated were:
\begin{itemize}
    \item \textbf{OpenAI \texttt{text-embedding-ada-002}:} A widely-used, high-performance proprietary model with a vector size of 1536. This model does not require any specific task instruction.
    \item \textbf{Nomic AI \texttt{nomic-embed-text}:} An open-source model with a vector size of 768. This model uses a prefix for different tasks. For our retrieval task, we used `search\_query:` for queries and `search\_document:` for documents.
    \item \textbf{Jina AI \texttt{jina-embeddings-v2-base-en}:} An open-source model with a vector size of 768. This model, like Ada-002, does not require a task instruction.
    \item \textbf{Jina AI \texttt{jina-embeddings-v3}:} A model that implements \textit{late chunking}, where chunks from the same document are tokenized together up to the model's context window of 8192 tokens, with a vector size of 1024. It requires a `task` argument, for which we used `retrieval.query` and `retrieval.passage` for queries and documents respectively.
    \item \textbf{Qwen \texttt{Qwen3-Embedding-0.6B}:} An open-source model with a vector size of 1024. This model benefits from a prompt for queries, for which we used the recommended `query` prompt name.
\end{itemize}
For vector storage and retrieval, we used Milvus \autocite{milvus}, a popular open-source vector database. We created an index of type HNSW (Hierarchical Navigable Small World) with the L2 distance metric. The index parameters were set to M=8 and efConstruction=64.

\subsection{Datasets}
The LongEmbed benchmark is composed of six datasets, two of which are synthetic and four are based on real-world data. The datasets are:
\begin{itemize}
    \item \textbf{2WikiMQA:} A multi-hop question-answering dataset.
    \item \textbf{NarrativeQA:} A question-answering dataset with long stories.
    \item \textbf{Needle In A Haystack (NIAH):} A synthetic dataset that tests the model's ability to find a specific piece of information ("needle") in a long text ("haystack").
    \item \textbf{Passkey Retrieval:} A synthetic dataset that tests the model's ability to retrieve a specific key from a long document.
    \item \textbf{QMSum:} A query-based meeting summarization dataset.
    \item \textbf{SummScreenFD:} A summarization dataset based on TV show scripts.
\end{itemize}

\subsection{Results and Discussion}
The retrieval performance of the five embedding models across the six datasets is detailed in Tables~\ref{tab:detailed_recall} through~\ref{tab:detailed_nDCG}, with an aggregated summary in Table~\ref{tab:summary}. The results not only highlight a clear winner within our long-context benchmark but also reflect a significant paradigm shift in the broader landscape of text embedding models.

\paragraph{The Rise of Compact, High-Performance Models}
While our results show \textbf{\texttt{Qwen3-Embedding-0.6B}} as the top performer, its true significance is revealed when contextualized within the global Massive Text Embedding Benchmark (MTEB) leaderboard. As of August 1st, 2025, this model ranks an impressive 4th overall, surpassed only by Google's proprietary \texttt{gemini-embedding-001} and its own, much larger 4B and 8B parameter siblings \autocite{mteb_leaderboard_2025}. With only 595M parameters, its performance is remarkable, demonstrating that architectural innovations and advanced training pipelines can allow smaller, open-source models to outperform far larger competitors. This efficiency is a direct result of its foundation on the Qwen3 architecture \autocite{yang2025qwen3technicalreport} and a sophisticated multi-stage training process that uses the foundation LLM itself to synthesize high-quality training data \autocite{zhang2025qwen3embeddingadvancingtext}.

\paragraph{A New Baseline: The Obsolescence of Ada-002}
For years, OpenAI's \textbf{\texttt{ada-002}} has been a de facto industry standard. However, these results, combined with its MTEB ranking of 167th, confirm that it has been decisively overcome. Newer models leveraging innovative architectures and specialized training techniques now offer vastly superior performance. The other models in our test also place significantly higher than \texttt{ada-002} on the MTEB leaderboard: \texttt{jina-embeddings-v3} ranks 24th, \texttt{nomic-embed-text} ranks 54th, and even the older \texttt{jina-embeddings-v2-base-en} ranks 154th \autocite{mteb_leaderboard_2025}. This trend signals that the benchmark for state-of-the-art has moved towards more open and efficient models.

\paragraph{The Enduring Value of Specialization}
Despite the dominance of a strong generalist model like Qwen3, our findings also underscore the continued importance of specialized models. \textbf{\texttt{jina-embeddings-v3}}, with its unique late chunking strategy, remains the top performer on the \texttt{passkey} task, proving the value of its architectural design for retrieving specific facts from structured long documents. Likewise, \textbf{\texttt{nomic-embed-text}}'s superior performance on the \texttt{needle} dataset confirms its status as a premier model for extreme "needle-in-a-haystack" scenarios. This demonstrates that for specific, challenging use cases, a specialized model can still be the optimal choice.

\paragraph{Distance Metrics and Model Compatibility}
For most models, the choice between L2 and Inner Product (IP) distance metrics yields comparable results, indicating that their embeddings are well-normalized. A critical observation, however, is the performance degradation of \textbf{\texttt{jina-embeddings-v2-base-en}} when using the IP metric. As seen across all tables, its performance collapses with IP; for example, its average Recall@5 drops from 51.85\% (L2) to a mere 12.12\% (IP). This underscores that its embedding vectors are not normalized for cosine similarity, making L2 distance the only viable choice for this model and highlighting the importance of aligning the retrieval metric with the model's output properties.

\paragraph{Conclusion}
In summary, this experiment illustrates a pivotal moment in the evolution of text embedding models. Our findings lead to the following conclusions:
\begin{itemize}
    \item The embedding landscape is no longer dominated by large, proprietary models. The exceptional performance of \textbf{\texttt{Qwen3-Embedding-0.6B}}, a compact 595M parameter model, signals a shift towards more efficient, accessible, and open-source solutions that achieve state-of-the-art results.
    \item Models like \textbf{\texttt{ada-002}}, while historically important, are now outdated. They have been surpassed by a new generation of models featuring more advanced architectures, training methodologies, and specialized strategies like late chunking.
    \item While general-purpose models have improved dramatically, \textbf{specialization remains highly valuable}. For niche but critical tasks like extreme long-context fact retrieval, models like \texttt{jina-embeddings-v3} and \texttt{nomic-embed-text} provide targeted performance that can exceed even the best generalists.
    \item \textbf{Technical implementation details are critical}. The choice of distance metric must be compatible with the model's properties, as the failure of \texttt{jina-embeddings-v2-base-en} with the IP metric demonstrates.
\end{itemize}

\begin{sidewaystable}[htbp!]
\centering
\footnotesize
\captionof{table}{Detailed RECALL Performance (in \%). Higher is better. Results are grouped by k-value. Best result per column within each k-value group is in \textbf{bold}.}
\label{tab:detailed_recall}
\begin{tabular}{llcccccccccccccc}
\toprule
\multirow{2}{*}{k} & \multirow{2}{*}{Model} & \multicolumn{2}{c}{2WikiMQA} & \multicolumn{2}{c}{NarrativeQA} & \multicolumn{2}{c}{NIAH} & \multicolumn{2}{c}{Passkey} & \multicolumn{2}{c}{QMSum} & \multicolumn{2}{c}{SummScreenFD} & \multicolumn{2}{c}{Average} \\
\cmidrule(lr){3-4} \cmidrule(lr){5-6} \cmidrule(lr){7-8} \cmidrule(lr){9-10} \cmidrule(lr){11-12} \cmidrule(lr){13-14} \cmidrule(lr){15-16}
& & L2 & IP & L2 & IP & L2 & IP & L2 & IP & L2 & IP & L2 & IP & L2 & IP \\
\midrule
\multirow{5}{*}{1} & Ada-002 & 79.33 & 81.00 & 54.44 & 54.27 & 2.00 & 1.50 & \textbf{13.25} & \textbf{14.50} & 45.65 & 45.91 & 86.90 & 86.90 & 52.01 & 51.98 \\
& Jina-V2 & \textbf{91.67} & 24.67 & 41.43 & 2.32 & 3.75 & 0.00 & 12.75 & 8.00 & 40.54 & 11.07 & 81.55 & 32.14 & 41.48 & 4.66 \\
& Jina-V3 & 90.67 & \textbf{90.67} & 57.54 & 57.46 & 1.25 & 0.75 & 10.50 & 12.50 & 47.22 & 47.22 & 84.23 & 84.23 & 54.69 & 54.67 \\
& Nomic & 67.33 & 67.33 & 47.51 & 48.15 & \textbf{8.75} & \textbf{9.25} & 5.75 & 3.75 & 21.02 & 21.02 & 52.98 & 52.98 & 42.67 & 43.13 \\
& Qwen3-0.6B & 89.67 & 89.67 & \textbf{62.99} & \textbf{62.85} & 4.25 & 3.75 & 9.50 & 13.75 & \textbf{50.49} & \textbf{50.49} & \textbf{88.10} & \textbf{88.10} & \textbf{59.45} & \textbf{59.45} \\
\midrule
\multirow{5}{*}{5} & Ada-002 & 87.33 & 88.00 & 63.85 & 63.64 & 4.25 & 4.00 & 32.50 & 34.00 & 65.82 & 66.34 & 96.73 & 96.73 & 62.71 & 62.66 \\
& Jina-V2 & 95.67 & 66.00 & 50.31 & 6.02 & 5.75 & 0.00 & 29.50 & 15.25 & 62.08 & 32.87 & 95.54 & 69.94 & 51.85 & 12.12 \\
& Jina-V3 & \textbf{96.67} & \textbf{96.67} & 67.40 & 67.42 & 2.00 & 2.00 & \textbf{40.50} & \textbf{44.50} & 67.58 & 67.58 & 97.92 & 97.92 & 66.09 & 66.22 \\
& Nomic & 79.67 & 79.67 & 58.64 & 59.33 & \textbf{14.50} & \textbf{14.75} & 9.25 & 5.25 & 39.03 & 39.03 & 82.74 & 82.74 & 54.69 & 55.11 \\
& Qwen3-0.6B & 95.67 & 95.67 & \textbf{73.27} & \textbf{73.16} & 5.50 & 6.50 & 15.75 & 40.75 & \textbf{72.30} & \textbf{72.30} & \textbf{98.21} & \textbf{98.21} & \textbf{70.55} & \textbf{71.23} \\
\midrule
\multirow{5}{*}{10} & Ada-002 & 90.00 & 90.67 & 66.35 & 66.22 & 5.25 & 5.00 & 38.75 & 41.00 & 73.94 & 74.46 & \textbf{98.81} & \textbf{98.81} & 65.91 & 65.94 \\
& Jina-V2 & 97.67 & 77.67 & 53.39 & 7.71 & 6.25 & 0.25 & 31.25 & 16.50 & 72.30 & 45.12 & 97.62 & 81.85 & 55.58 & 15.43 \\
& Jina-V3 & \textbf{98.33} & \textbf{98.33} & 70.67 & 70.71 & 2.25 & 2.50 & \textbf{53.50} & \textbf{58.75} & 75.05 & 75.05 & \textbf{98.81} & \textbf{98.81} & 69.94 & 70.13 \\
& Nomic & 85.00 & 85.00 & 62.32 & 62.88 & \textbf{17.25} & \textbf{17.50} & 9.25 & 5.25 & 47.35 & 47.35 & 91.07 & 91.07 & 58.92 & 59.24 \\
& Qwen3-0.6B & 98.00 & 98.00 & \textbf{76.48} & \textbf{76.18} & 6.25 & 7.00 & 16.00 & 42.25 & \textbf{79.96} & \textbf{79.96} & 98.21 & 98.21 & \textbf{74.00} & \textbf{74.58} \\
\midrule
\multirow{5}{*}{25} & Ada-002 & 92.33 & 92.67 & 69.44 & 69.45 & 7.50 & 7.50 & 49.75 & 53.25 & 82.25 & 82.91 & \textbf{99.70} & \textbf{99.70} & 69.74 & 69.93 \\
& Jina-V2 & 98.67 & 91.67 & 56.97 & 11.65 & 8.50 & 1.50 & 33.25 & 17.75 & 81.99 & 64.37 & 97.92 & 93.45 & 59.63 & 21.37 \\
& Jina-V3 & \textbf{99.33} & \textbf{99.33} & 74.30 & 74.28 & 5.25 & 5.25 & \textbf{54.25} & \textbf{60.00} & 83.76 & 83.76 & \textbf{99.70} & \textbf{99.70} & 73.92 & 74.08 \\
& Nomic & 89.33 & 89.33 & 66.78 & 67.31 & \textbf{20.00} & \textbf{19.75} & 9.25 & 5.25 & 62.41 & 62.41 & 96.13 & 96.13 & 64.41 & 64.70 \\
& Qwen3-0.6B & 98.33 & 98.33 & \textbf{79.88} & \textbf{79.71} & 8.25 & 11.25 & 16.00 & 42.25 & \textbf{89.00} & \textbf{89.00} & \textbf{99.70} & \textbf{99.70} & \textbf{77.79} & \textbf{78.53} \\
\bottomrule
\end{tabular}
\end{sidewaystable}
\clearpage
\begin{sidewaystable}[htbp!]
\centering
\footnotesize
\captionof{table}{Detailed PRECISION Performance (in \%). Higher is better. Results are grouped by k-value. Best result per column within each k-value group is in \textbf{bold}.}
\label{tab:detailed_precision}
\begin{tabular}{llcccccccccccccc}
\toprule
\multirow{2}{*}{k} & \multirow{2}{*}{Model} & \multicolumn{2}{c}{2WikiMQA} & \multicolumn{2}{c}{NarrativeQA} & \multicolumn{2}{c}{NIAH} & \multicolumn{2}{c}{Passkey} & \multicolumn{2}{c}{QMSum} & \multicolumn{2}{c}{SummScreenFD} & \multicolumn{2}{c}{Average} \\
\cmidrule(lr){3-4} \cmidrule(lr){5-6} \cmidrule(lr){7-8} \cmidrule(lr){9-10} \cmidrule(lr){11-12} \cmidrule(lr){13-14} \cmidrule(lr){15-16}
& & L2 & IP & L2 & IP & L2 & IP & L2 & IP & L2 & IP & L2 & IP & L2 & IP \\
\midrule
\multirow{5}{*}{1} & Ada-002 & 79.33 & 81.00 & 54.44 & 54.27 & 2.00 & 1.50 & \textbf{13.25} & \textbf{14.50} & 45.65 & 45.91 & 86.90 & 86.90 & 52.01 & 51.98 \\
& Jina-V2 & \textbf{91.67} & 24.67 & 41.43 & 2.32 & 3.75 & 0.00 & 12.75 & 8.00 & 40.54 & 11.07 & 81.55 & 32.14 & 41.48 & 4.66 \\
& Jina-V3 & 90.67 & \textbf{90.67} & 57.54 & 57.46 & 1.25 & 0.75 & 10.50 & 12.50 & 47.22 & 47.22 & 84.23 & 84.23 & 54.69 & 54.67 \\
& Nomic & 67.33 & 67.33 & 47.51 & 48.15 & \textbf{8.75} & \textbf{9.25} & 5.75 & 3.75 & 21.02 & 21.02 & 52.98 & 52.98 & 42.67 & 43.13 \\
& Qwen3-0.6B & 89.67 & 89.67 & \textbf{62.99} & \textbf{62.85} & 4.25 & 3.75 & 9.50 & 13.75 & \textbf{50.49} & \textbf{50.49} & \textbf{88.10} & \textbf{88.10} & \textbf{59.45} & \textbf{59.45} \\
\midrule
\multirow{5}{*}{5} & Ada-002 & 17.47 & 17.60 & 12.77 & 12.73 & 0.85 & 0.80 & 6.50 & 6.80 & 13.16 & 13.27 & 19.35 & 19.35 & 12.54 & 12.53 \\
& Jina-V2 & 19.13 & 13.20 & 10.06 & 1.20 & 1.15 & 0.00 & 5.90 & 3.05 & 12.42 & 6.57 & 19.11 & 13.99 & 10.37 & 2.42 \\
& Jina-V3 & \textbf{19.33} & \textbf{19.33} & 13.48 & 13.48 & 0.40 & 0.40 & \textbf{8.10} & \textbf{8.90} & 13.52 & 13.52 & 19.58 & 19.58 & 13.22 & 13.24 \\
& Nomic & 15.93 & 15.93 & 11.73 & 11.87 & \textbf{2.90} & \textbf{2.95} & 1.85 & 1.05 & 7.81 & 7.81 & 16.55 & 16.55 & 10.94 & 11.02 \\
& Qwen3-0.6B & 19.13 & 19.13 & \textbf{14.65} & \textbf{14.63} & 1.10 & 1.30 & 3.15 & 8.15 & \textbf{14.46} & \textbf{14.46} & \textbf{19.64} & \textbf{19.64} & \textbf{14.11} & \textbf{14.25} \\
\midrule
\multirow{5}{*}{10} & Ada-002 & 9.00 & 9.07 & 6.64 & 6.62 & 0.53 & 0.50 & 3.87 & 4.10 & 7.39 & 7.45 & \textbf{9.88} & \textbf{9.88} & 6.59 & 6.59 \\
& Jina-V2 & 9.77 & 7.77 & 5.34 & 0.77 & 0.63 & 0.03 & 3.12 & 1.65 & 7.23 & 4.51 & 9.76 & 8.18 & 5.56 & 1.54 \\
& Jina-V3 & \textbf{9.83} & \textbf{9.83} & 7.07 & 7.07 & 0.22 & 0.25 & \textbf{5.35} & \textbf{5.88} & 7.50 & 7.50 & \textbf{9.88} & \textbf{9.88} & 6.99 & 7.01 \\
& Nomic & 8.50 & 8.50 & 6.23 & 6.29 & \textbf{1.72} & \textbf{1.75} & 0.93 & 0.53 & 4.73 & 4.73 & 9.11 & 9.11 & 5.89 & 5.92 \\
& Qwen3-0.6B & 9.80 & 9.80 & \textbf{7.65} & \textbf{7.62} & 0.63 & 0.70 & 1.60 & 4.22 & \textbf{8.00} & \textbf{8.00} & 9.82 & 9.82 & \textbf{7.40} & \textbf{7.46} \\
\midrule
\multirow{5}{*}{25} & Ada-002 & 3.69 & 3.71 & 2.78 & 2.78 & 0.30 & 0.30 & 1.99 & 2.13 & 3.29 & 3.32 & \textbf{3.99} & \textbf{3.99} & 2.79 & 2.80 \\
& Jina-V2 & 3.95 & 3.67 & 2.28 & 0.47 & 0.34 & 0.06 & 1.33 & 0.71 & 3.28 & 2.57 & 3.92 & 3.74 & 2.39 & 0.85 \\
& Jina-V3 & \textbf{3.97} & \textbf{3.97} & 2.97 & 2.97 & 0.21 & 0.21 & \textbf{2.17} & \textbf{2.40} & 3.35 & 3.35 & \textbf{3.99} & \textbf{3.99} & 2.96 & 2.96 \\
& Nomic & 3.57 & 3.57 & 2.67 & 2.69 & \textbf{0.80} & \textbf{0.79} & 0.37 & 0.21 & 2.50 & 2.50 & 3.85 & 3.85 & 2.58 & 2.59 \\
& Qwen3-0.6B & 3.93 & 3.93 & \textbf{3.20} & \textbf{3.19} & 0.33 & 0.45 & 0.64 & 1.69 & \textbf{3.56} & \textbf{3.56} & \textbf{3.99} & \textbf{3.99} & \textbf{3.11} & \textbf{3.14} \\
\bottomrule
\end{tabular}
\end{sidewaystable}
\clearpage
\begin{sidewaystable}[htbp!]
\centering
\footnotesize
\captionof{table}{Detailed MRR Performance (in \%). Higher is better. Results are grouped by k-value. Best result per column within each k-value group is in \textbf{bold}.}
\label{tab:detailed_mrr}
\begin{tabular}{llcccccccccccccc}
\toprule
\multirow{2}{*}{k} & \multirow{2}{*}{Model} & \multicolumn{2}{c}{2WikiMQA} & \multicolumn{2}{c}{NarrativeQA} & \multicolumn{2}{c}{NIAH} & \multicolumn{2}{c}{Passkey} & \multicolumn{2}{c}{QMSum} & \multicolumn{2}{c}{SummScreenFD} & \multicolumn{2}{c}{Average} \\
\cmidrule(lr){3-4} \cmidrule(lr){5-6} \cmidrule(lr){7-8} \cmidrule(lr){9-10} \cmidrule(lr){11-12} \cmidrule(lr){13-14} \cmidrule(lr){15-16}
& & L2 & IP & L2 & IP & L2 & IP & L2 & IP & L2 & IP & L2 & IP & L2 & IP \\
\midrule
\multirow{5}{*}{1} & Ada-002 & 79.33 & 81.00 & 54.44 & 54.27 & 2.00 & 1.50 & \textbf{13.25} & \textbf{14.50} & 45.65 & 45.91 & 86.90 & 86.90 & 52.01 & 51.98 \\
& Jina-V2 & \textbf{91.67} & 24.67 & 41.43 & 2.32 & 3.75 & 0.00 & 12.75 & 8.00 & 40.54 & 11.07 & 81.55 & 32.14 & 41.48 & 4.66 \\
& Jina-V3 & 90.67 & \textbf{90.67} & 57.54 & 57.46 & 1.25 & 0.75 & 10.50 & 12.50 & 47.22 & 47.22 & 84.23 & 84.23 & 54.69 & 54.67 \\
& Nomic & 67.33 & 67.33 & 47.51 & 48.15 & \textbf{8.75} & \textbf{9.25} & 5.75 & 3.75 & 21.02 & 21.02 & 52.98 & 52.98 & 42.67 & 43.13 \\
& Qwen3-0.6B & 89.67 & 89.67 & \textbf{62.99} & \textbf{62.85} & 4.25 & 3.75 & 9.50 & 13.75 & \textbf{50.49} & \textbf{50.49} & \textbf{88.10} & \textbf{88.10} & \textbf{59.45} & \textbf{59.45} \\
\midrule
\multirow{5}{*}{5} & Ada-002 & 82.28 & 83.57 & 58.14 & 57.90 & 2.81 & 2.49 & 20.36 & 21.56 & 53.30 & 53.62 & 91.11 & 91.11 & 56.17 & 56.08 \\
& Jina-V2 & 93.13 & 39.76 & 44.87 & 3.62 & 4.55 & 0.00 & 19.76 & 10.88 & 48.47 & 18.67 & 87.17 & 46.11 & 45.47 & 7.31 \\
& Jina-V3 & \textbf{93.22} & \textbf{93.22} & 61.43 & 61.34 & 1.46 & 1.21 & \textbf{20.45} & 23.56 & 54.77 & 54.77 & 90.10 & 90.10 & 59.09 & 59.11 \\
& Nomic & 72.21 & 72.21 & 51.83 & 52.49 & \textbf{11.28} & \textbf{11.63} & 7.20 & 4.46 & 27.64 & 27.64 & 64.18 & 64.18 & 47.30 & 47.75 \\
& Qwen3-0.6B & 92.28 & 92.28 & \textbf{67.02} & \textbf{66.88} & 4.83 & 4.78 & 12.18 & \textbf{24.09} & \textbf{58.59} & \textbf{58.59} & \textbf{92.31} & \textbf{92.31} & \textbf{63.77} & \textbf{64.02} \\
\midrule
\multirow{5}{*}{10} & Ada-002 & 82.68 & 83.96 & 58.48 & 58.25 & 2.94 & 2.61 & 21.22 & 22.54 & 54.40 & 54.73 & 91.39 & 91.39 & 56.61 & 56.53 \\
& Jina-V2 & 93.42 & 41.35 & 45.29 & 3.84 & 4.63 & 0.03 & 20.01 & 11.04 & 49.84 & 20.31 & 87.45 & 47.74 & 45.97 & 7.76 \\
& Jina-V3 & \textbf{93.44} & \textbf{93.44} & 61.87 & 61.78 & 1.49 & 1.27 & \textbf{22.38} & \textbf{25.70} & 55.79 & 55.79 & 90.23 & 90.23 & 59.61 & 59.64 \\
& Nomic & 72.96 & 72.96 & 52.33 & 52.97 & \textbf{11.66} & \textbf{12.01} & 7.20 & 4.46 & 28.74 & 28.74 & 65.37 & 65.37 & 47.87 & 48.30 \\
& Qwen3-0.6B & 92.60 & 92.60 & \textbf{67.45} & \textbf{67.29} & 4.92 & 4.84 & 12.22 & 24.33 & \textbf{59.65} & \textbf{59.65} & \textbf{92.31} & \textbf{92.31} & \textbf{64.24} & \textbf{64.47} \\
\midrule
\multirow{5}{*}{25} & Ada-002 & 82.83 & 84.08 & 58.68 & 58.46 & 3.07 & 2.75 & 21.91 & 23.29 & 54.93 & 55.27 & 91.44 & 91.44 & 56.86 & 56.79 \\
& Jina-V2 & 93.48 & 42.27 & 45.52 & 4.08 & 4.75 & 0.10 & 20.12 & 11.12 & 50.45 & 21.52 & 87.47 & 48.52 & 46.23 & 8.12 \\
& Jina-V3 & \textbf{93.52} & \textbf{93.52} & 62.10 & 62.01 & 1.68 & 1.43 & \textbf{22.42} & \textbf{25.78} & 56.35 & 56.35 & 90.28 & 90.28 & 59.87 & 59.89 \\
& Nomic & 73.21 & 73.21 & 52.61 & 53.24 & \textbf{11.86} & \textbf{12.17} & 7.20 & 4.46 & 29.69 & 29.69 & 65.69 & 65.69 & 48.22 & 48.64 \\
& Qwen3-0.6B & 92.62 & 92.62 & \textbf{67.67} & \textbf{67.52} & 5.05 & 5.13 & 12.22 & 24.33 & \textbf{60.21} & \textbf{60.21} & \textbf{92.41} & \textbf{92.41} & \textbf{64.48} & \textbf{64.73} \\
\bottomrule
\end{tabular}
\end{sidewaystable}
\clearpage
\begin{sidewaystable}[htbp!]
\centering
\footnotesize
\captionof{table}{Detailed NDCG Performance (in \%). Higher is better. Results are grouped by k-value. Best result per column within each k-value group is in \textbf{bold}.}
\label{tab:detailed_nDCG}
\begin{tabular}{llcccccccccccccc}
\toprule
\multirow{2}{*}{k} & \multirow{2}{*}{Model} & \multicolumn{2}{c}{2WikiMQA} & \multicolumn{2}{c}{NarrativeQA} & \multicolumn{2}{c}{NIAH} & \multicolumn{2}{c}{Passkey} & \multicolumn{2}{c}{QMSum} & \multicolumn{2}{c}{SummScreenFD} & \multicolumn{2}{c}{Average} \\
\cmidrule(lr){3-4} \cmidrule(lr){5-6} \cmidrule(lr){7-8} \cmidrule(lr){9-10} \cmidrule(lr){11-12} \cmidrule(lr){13-14} \cmidrule(lr){15-16}
& & L2 & IP & L2 & IP & L2 & IP & L2 & IP & L2 & IP & L2 & IP & L2 & IP \\
\midrule
\multirow{5}{*}{1} & Ada-002 & 79.33 & 81.00 & 54.44 & 54.27 & 2.00 & 1.50 & \textbf{13.25} & \textbf{14.50} & 45.65 & 45.91 & 86.90 & 86.90 & 52.01 & 51.98 \\
& Jina-V2 & \textbf{91.67} & 24.67 & 41.43 & 2.32 & 3.75 & 0.00 & 12.75 & 8.00 & 40.54 & 11.07 & 81.55 & 32.14 & 41.48 & 4.66 \\
& Jina-V3 & 90.67 & \textbf{90.67} & 57.54 & 57.46 & 1.25 & 0.75 & 10.50 & 12.50 & 47.22 & 47.22 & 84.23 & 84.23 & 54.69 & 54.67 \\
& Nomic & 67.33 & 67.33 & 47.51 & 48.15 & \textbf{8.75} & \textbf{9.25} & 5.75 & 3.75 & 21.02 & 21.02 & 52.98 & 52.98 & 42.67 & 43.13 \\
& Qwen3-0.6B & 89.67 & 89.67 & \textbf{62.99} & \textbf{62.85} & 4.25 & 3.75 & 9.50 & 13.75 & \textbf{50.49} & \textbf{50.49} & \textbf{88.10} & \textbf{88.10} & \textbf{59.45} & \textbf{59.45} \\
\midrule
\multirow{5}{*}{5} & Ada-002 & 83.53 & 84.68 & 59.57 & 59.34 & 3.16 & 2.87 & 23.38 & 24.65 & 56.42 & 56.79 & 92.54 & 92.54 & 57.81 & 57.73 \\
& Jina-V2 & 93.76 & 46.27 & 46.23 & 4.21 & 4.85 & 0.00 & 22.23 & 11.98 & 51.86 & 22.19 & 89.28 & 52.05 & 47.07 & 8.50 \\
& Jina-V3 & \textbf{94.09} & \textbf{94.09} & 62.93 & 62.87 & 1.59 & 1.41 & \textbf{25.38} & \textbf{28.73} & 57.96 & 57.96 & 92.10 & 92.10 & 60.84 & 60.89 \\
& Nomic & 74.07 & 74.07 & 53.54 & 54.21 & \textbf{12.10} & \textbf{12.43} & 7.72 & 4.66 & 30.48 & 30.48 & 68.81 & 68.81 & 49.15 & 49.59 \\
& Qwen3-0.6B & 93.14 & 93.14 & \textbf{68.59} & \textbf{68.46} & 5.01 & 5.21 & 13.09 & 28.26 & \textbf{62.01} & \textbf{62.01} & \textbf{93.80} & \textbf{93.80} & \textbf{65.47} & \textbf{65.82} \\
\midrule
\multirow{5}{*}{10} & Ada-002 & 84.45 & 85.57 & 60.39 & 60.18 & 3.48 & 3.18 & 25.43 & 26.96 & 59.07 & 59.44 & 93.22 & 93.22 & 58.86 & 58.80 \\
& Jina-V2 & 94.43 & 50.08 & 47.24 & 4.76 & 5.03 & 0.07 & 22.82 & 12.37 & 55.17 & 26.15 & 89.95 & 55.95 & 48.28 & 9.58 \\
& Jina-V3 & \textbf{94.63} & \textbf{94.63} & 63.99 & 63.93 & 1.67 & 1.56 & \textbf{29.77} & \textbf{33.58} & 60.40 & 60.40 & 92.39 & 92.39 & 62.10 & 62.16 \\
& Nomic & 75.84 & 75.84 & 54.74 & 55.35 & \textbf{13.01} & \textbf{13.32} & 7.72 & 4.66 & 33.15 & 33.15 & 71.58 & 71.58 & 50.53 & 50.93 \\
& Qwen3-0.6B & 93.91 & 93.91 & \textbf{69.63} & \textbf{69.44} & 5.23 & 5.37 & 13.17 & 28.79 & \textbf{64.52} & \textbf{64.52} & \textbf{93.80} & \textbf{93.80} & \textbf{66.59} & \textbf{66.91} \\
\midrule
\multirow{5}{*}{25} & Ada-002 & 85.01 & 86.06 & 61.15 & 60.97 & 4.02 & 3.77 & 28.11 & 29.93 & 61.11 & 61.52 & 93.43 & 93.43 & 59.80 & 59.78 \\
& Jina-V2 & 94.67 & 53.55 & 48.11 & 5.70 & 5.55 & 0.37 & 23.28 & 12.69 & 57.53 & 30.86 & 90.03 & 58.85 & 49.27 & 11.02 \\
& Jina-V3 & \textbf{94.90} & \textbf{94.90} & 64.89 & 64.81 & 2.40 & 2.21 & \textbf{29.95} & \textbf{33.88} & 62.54 & 62.54 & 92.61 & 92.61 & 63.08 & 63.13 \\
& Nomic & 76.88 & 76.88 & 55.83 & 56.43 & \textbf{13.71} & \textbf{13.90} & 7.72 & 4.66 & 36.83 & 36.83 & 72.83 & 72.83 & 51.87 & 52.26 \\
& Qwen3-0.6B & 93.99 & 93.99 & \textbf{70.47} & \textbf{70.31} & 5.73 & 6.43 & 13.17 & 28.79 & \textbf{66.71} & \textbf{66.71} & \textbf{94.18} & \textbf{94.18} & \textbf{67.52} & \textbf{67.89} \\
\bottomrule
\end{tabular}
\end{sidewaystable}
\clearpage
\begin{table}[htbp!]
\centering
\footnotesize
\caption{Summary of Average Performance Across All Datasets (in \%). Best result per column within each k-group is in \textbf{{bold}}.}
\label{tab:summary}
\begin{tabular}{llcccccccc}
\toprule
\multirow{2}{*}{k} & \multirow{2}{*}{Model} & \multicolumn{2}{c}{RECALL} & \multicolumn{2}{c}{PRECISION} & \multicolumn{2}{c}{MRR} & \multicolumn{2}{c}{NDCG} \\
\cmidrule(lr){3-4} \cmidrule(lr){5-6} \cmidrule(lr){7-8} \cmidrule(lr){9-10}
& & L2 & IP & L2 & IP & L2 & IP & L2 & IP \\
\midrule
\multirow{5}{*}{1} & Ada-002 & 52.01 & 51.98 & 52.01 & 51.98 & 52.01 & 51.98 & 52.01 & 51.98 \\
& Jina-V2 & 41.48 & 4.66 & 41.48 & 4.66 & 41.48 & 4.66 & 41.48 & 4.66 \\
& Jina-V3 & 54.69 & 54.67 & 54.69 & 54.67 & 54.69 & 54.67 & 54.69 & 54.67 \\
& Nomic & 42.67 & 43.13 & 42.67 & 43.13 & 42.67 & 43.13 & 42.67 & 43.13 \\
& Qwen3-0.6B & \textbf{59.45} & \textbf{59.45} & \textbf{59.45} & \textbf{59.45} & \textbf{59.45} & \textbf{59.45} & \textbf{59.45} & \textbf{59.45} \\
\midrule
\multirow{5}{*}{5} & Ada-002 & 62.71 & 62.66 & 12.54 & 12.53 & 56.17 & 56.08 & 57.81 & 57.73 \\
& Jina-V2 & 51.85 & 12.12 & 10.37 & 2.42 & 45.47 & 7.31 & 47.07 & 8.50 \\
& Jina-V3 & 66.09 & 66.22 & 13.22 & 13.24 & 59.09 & 59.11 & 60.84 & 60.89 \\
& Nomic & 54.69 & 55.11 & 10.94 & 11.02 & 47.30 & 47.75 & 49.15 & 49.59 \\
& Qwen3-0.6B & \textbf{70.55} & \textbf{71.23} & \textbf{14.11} & \textbf{14.25} & \textbf{63.77} & \textbf{64.02} & \textbf{65.47} & \textbf{65.82} \\
\midrule
\multirow{5}{*}{10} & Ada-002 & 65.91 & 65.94 & 6.59 & 6.59 & 56.61 & 56.53 & 58.86 & 58.80 \\
& Jina-V2 & 55.58 & 15.43 & 5.56 & 1.54 & 45.97 & 7.76 & 48.28 & 9.58 \\
& Jina-V3 & 69.94 & 70.13 & 6.99 & 7.01 & 59.61 & 59.64 & 62.10 & 62.16 \\
& Nomic & 58.92 & 59.24 & 5.89 & 5.92 & 47.87 & 48.30 & 50.53 & 50.93 \\
& Qwen3-0.6B & \textbf{74.00} & \textbf{74.58} & \textbf{7.40} & \textbf{7.46} & \textbf{64.24} & \textbf{64.47} & \textbf{66.59} & \textbf{66.91} \\
\midrule
\multirow{5}{*}{25} & Ada-002 & 69.74 & 69.93 & 2.79 & 2.80 & 56.86 & 56.79 & 59.80 & 59.78 \\
& Jina-V2 & 59.63 & 21.37 & 2.39 & 0.85 & 46.23 & 8.12 & 49.27 & 11.02 \\
& Jina-V3 & 73.92 & 74.08 & 2.96 & 2.96 & 59.87 & 59.89 & 63.08 & 63.13 \\
& Nomic & 64.41 & 64.70 & 2.58 & 2.59 & 48.22 & 48.64 & 51.87 & 52.26 \\
& Qwen3-0.6B & \textbf{77.79} & \textbf{78.53} & \textbf{3.11} & \textbf{3.14} & \textbf{64.48} & \textbf{64.73} & \textbf{67.52} & \textbf{67.89} \\
\bottomrule
\end{tabular}
\end{table}

\section{Experiment 2: Reranking Strategies}
\label{sec:exp_reranking}
This experiment evaluated the benefit of adding a reranking step after the initial retrieval. A reranker, typically a cross-encoder model, re-evaluates the top-k documents returned by the initial retriever, providing a more accurate relevance score. This allows for the rescue of relevant documents that may have been ranked lower by the initial semantic search.

\subsection{Reranking Models Compared}
We compared three cross-encoder models to rerank the top 500 candidates returned by the similarity search:
\begin{itemize}
    \item \textbf{No Reranker (Baseline):} The initial retrieval order is used.
    \item \textbf{GTE ML Reranker Base:} A powerful reranker based on the DeBERTa-v3 architecture, known for its strong performance on various reranking tasks \autocite{li2024generative}.
    \item \textbf{Jina Reranker v1 Tiny EN:} A lightweight and efficient reranker from Jina AI, designed for speed and scalability \autocite{jina-reranker-v1-tiny-en}.
    \item \textbf{MXBAI Rerank Base v2:} A multilingual reranker that has shown competitive performance on diverse datasets \autocite{pradeep2024reranking}.
\end{itemize}

\subsection{Results and Discussion}
\begin{table*}[htbp]
\centering
\footnotesize
\setlength{\tabcolsep}{3pt}
\caption{Comparison of Reranker Performance Across Different Chunking Strategies and Thresholding Methods. The highest score in each metric column is highlighted in bold.}
\label{tab:reranker_performance}
\begin{tabular}{llrrrrrr}
\toprule
 &  & \multicolumn{3}{c}{Full Page} & \multicolumn{3}{c}{Page Split} \\
 Threshold & Model & F1 Score & Precision & Recall & F1 Score & Precision & Recall \\
\midrule
\multirow[t]{4}{*}{Baseline} & No Reranker & 0.021 & 0.011 & 0.967 & 0.021 & 0.011 & 0.918 \\
 & GTE ML Reranker Base & 0.024 & 0.012 & 0.871 & 0.027 & 0.014 & 0.840 \\
 & Jina Reranker v1 Tiny EN & 0.023 & 0.012 & 0.919 & 0.021 & 0.011 & 0.891 \\
 & MXBAI Rerank Base v2 & 0.062 & 0.033 & 0.947 & 0.060 & 0.032 & 0.880 \\
\cline{1-8}
\multirow[t]{4}{*}{GMM} & No Reranker & 0.016 & 0.008 & 0.986 & 0.014 & 0.007 & 0.920 \\
 & GTE ML Reranker Base & 0.008 & 0.004 & \textbf{0.988} & 0.006 & 0.003 & 0.924 \\
 & Jina Reranker v1 Tiny EN & 0.023 & 0.012 & 0.976 & 0.020 & 0.010 & \textbf{0.924} \\
 & MXBAI Rerank Base v2 & 0.009 & 0.005 & 0.986 & 0.008 & 0.004 & 0.920 \\
\cline{1-8}
\multirow[t]{4}{*}{Knee} & No Reranker & 0.082 & 0.044 & 0.881 & 0.081 & 0.043 & 0.878 \\
 & GTE ML Reranker Base & 0.108 & 0.059 & 0.936 & 0.095 & 0.052 & 0.871 \\
 & Jina Reranker v1 Tiny EN & 0.107 & 0.060 & 0.838 & 0.090 & 0.049 & 0.862 \\
 & MXBAI Rerank Base v2 & 0.105 & 0.066 & 0.388 & 0.092 & 0.059 & 0.347 \\
\cline{1-8}
\multirow[t]{4}{*}{Max Gap} & No Reranker & 0.456 & 0.404 & 0.681 & 0.598 & 0.548 & 0.751 \\
 & GTE ML Reranker Base & \textbf{0.654} & \textbf{0.594} & 0.867 & \textbf{0.646} & \textbf{0.601} & 0.797 \\
 & Jina Reranker v1 Tiny EN & 0.435 & 0.379 & 0.893 & 0.468 & 0.410 & 0.800 \\
 & MXBAI Rerank Base v2 & 0.071 & 0.052 & 0.986 & 0.055 & 0.037 & 0.918 \\
\cline{1-8}
\multirow[t]{4}{*}{Otsu} & No Reranker & 0.021 & 0.011 & 0.986 & 0.017 & 0.008 & 0.918 \\
 & GTE ML Reranker Base & 0.018 & 0.009 & \textbf{0.988} & 0.009 & 0.005 & 0.922 \\
 & Jina Reranker v1 Tiny EN & 0.035 & 0.019 & 0.979 & 0.033 & 0.018 & 0.920 \\
 & MXBAI Rerank Base v2 & 0.012 & 0.006 & 0.986 & 0.011 & 0.006 & 0.922 \\
\cline{1-8}
\multirow[t]{4}{*}{Percentile} & No Reranker & 0.037 & 0.019 & 0.952 & 0.036 & 0.018 & 0.907 \\
 & GTE ML Reranker Base & 0.038 & 0.019 & 0.974 & 0.035 & 0.018 & 0.900 \\
 & Jina Reranker v1 Tiny EN & 0.038 & 0.019 & 0.945 & 0.035 & 0.018 & 0.902 \\
 & MXBAI Rerank Base v2 & 0.037 & 0.019 & 0.971 & 0.035 & 0.018 & 0.902 \\
\cline{1-8}
\multirow[t]{4}{*}{2nd Derivative} & No Reranker & 0.254 & 0.199 & 0.721 & 0.262 & 0.198 & 0.776 \\
 & GTE ML Reranker Base & 0.305 & 0.237 & 0.845 & 0.266 & 0.203 & 0.806 \\
 & Jina Reranker v1 Tiny EN & 0.222 & 0.177 & 0.881 & 0.235 & 0.184 & 0.820 \\
 & MXBAI Rerank Base v2 & 0.059 & 0.042 & 0.981 & 0.049 & 0.035 & 0.916 \\
\cline{1-8}
\bottomrule
\end{tabular}
\end{table*}

The results in Table \ref{tab:reranker_performance} and the visualizations in Figure \ref{fig:avg_metrics_faceted} reveal a complex interplay between reranking models, chunking strategies, and thresholding methods. The key takeaway is that the effectiveness of a reranker is not absolute but is highly dependent on the subsequent thresholding strategy used to select the final documents.

When no intelligent thresholding is applied (Baseline) or when suboptimal methods like GMM or Otsu are used, the performance gains from reranking are minimal or even negative. These methods often pass too many documents (as seen in Table \ref{tab:merged_retrieval_stats}), diluting the context with noise and negating the precision benefits of the reranker. For instance, with the Baseline threshold, the F1 scores remain low (0.021-0.062) across all models because the high recall is undermined by extremely low precision.

However, the true power of reranking is unlocked when combined with an effective thresholding method like \textbf{Max Gap}. With this method, the \textbf{GTE ML Reranker Base} model achieves the highest F1 score (\textbf{0.654} for Full Page, \textbf{0.646} for Page Split), demonstrating a dramatic improvement over the No Reranker baseline (0.456 and 0.598). This is because the Max Gap method is highly effective at identifying the point of diminishing returns in the reranked list, selecting a small, highly relevant set of documents. As shown in Table \ref{tab:merged_retrieval_stats}, the GTE reranker with Max Gap passes an average of only 4.76 items, leading to a high precision of 0.594.

Conversely, the \textbf{MXBAI Rerank Base v2} model shows poor performance with aggressive thresholding methods like Knee and Max Gap. This is because the model's score distribution is very flat, making it difficult for these methods to find a clear cutoff point. As a result, it passes too few documents (e.g., only 4.38 items with Knee), leading to a collapse in recall (0.388) and a low F1 score (0.105).

In conclusion, this experiment demonstrates that simply adding a reranker is not enough. The combination of a powerful reranker like \textbf{GTE ML Reranker Base} with an aggressive, well-suited thresholding method like \textbf{Max Gap} is what yields the best performance, maximizing precision without sacrificing essential recall.
\begin{table*}[htbp]
\centering
\footnotesize
\caption{Comparison of the average number of items and unique documents passed by each thresholding method, faceted by chunking strategy and reranker model. Lower is better. The lowest value in each column is highlighted in bold.}
\label{tab:merged_retrieval_stats}
\begin{tabular}{llrrrr}
\toprule
 & Chunking & \multicolumn{2}{c}{Full Page} & \multicolumn{2}{c}{Page Split} \\
 &  & Docs Passed & Items Passed & Docs Passed & Items Passed \\
Threshold & Model &  &  &  &  \\
\midrule
\multirow[t]{4}{*}{Baseline} & No Reranker & 7.13 & 104.00 & 5.43 & 101.14 \\
 & GTE ML Reranker Base & 6.28 & 79.70 & 5.46 & 68.79 \\
 & Jina Reranker v1 Tiny EN & 7.13 & 94.76 & 5.91 & 103.57 \\
 & MXBAI Rerank Base v2 & 6.40 & 102.88 & 5.42 & 109.06 \\
\cline{1-6}
\multirow[t]{4}{*}{GMM} & No Reranker & 7.82 & 158.96 & 6.54 & 161.30 \\
 & GTE ML Reranker Base & 11.97 & 287.73 & 10.73 & 340.32 \\
 & Jina Reranker v1 Tiny EN & 7.82 & 164.49 & 6.71 & 159.13 \\
 & MXBAI Rerank Base v2 & 10.02 & 242.08 & 8.65 & 247.35 \\
\cline{1-6}
\multirow[t]{4}{*}{Knee} & No Reranker & 3.17 & 26.28 & 2.70 & 25.82 \\
 & GTE ML Reranker Base & 3.58 & 21.39 & 3.23 & 22.02 \\
 & Jina Reranker v1 Tiny EN & 3.26 & 20.05 & 3.16 & 25.56 \\
 & MXBAI Rerank Base v2 & \textbf{1.06} & \textbf{4.38} & \textbf{1.11} & 4.50 \\
\cline{1-6}
\multirow[t]{4}{*}{Max Gap} & No Reranker & 1.39 & 4.68 & 1.12 & \textbf{2.36} \\
 & GTE ML Reranker Base & 1.40 & 4.76 & 1.22 & 11.87 \\
 & Jina Reranker v1 Tiny EN & 2.99 & 93.73 & 1.69 & 34.14 \\
 & MXBAI Rerank Base v2 & 8.10 & 213.62 & 6.49 & 185.26 \\
\cline{1-6}
\multirow[t]{4}{*}{Otsu} & No Reranker & 6.16 & 126.08 & 5.20 & 128.48 \\
 & GTE ML Reranker Base & 8.70 & 175.23 & 8.99 & 232.65 \\
 & Jina Reranker v1 Tiny EN & 6.75 & 139.35 & 5.56 & 137.89 \\
 & MXBAI Rerank Base v2 & 9.86 & 231.67 & 7.88 & 210.58 \\
\cline{1-6}
\multirow[t]{4}{*}{Percentile} & No Reranker & 4.33 & 50.02 & 3.53 & 50.00 \\
 & GTE ML Reranker Base & 5.12 & 50.12 & 4.80 & 50.12 \\
 & Jina Reranker v1 Tiny EN & 5.24 & 49.71 & 4.30 & 50.01 \\
 & MXBAI Rerank Base v2 & 4.55 & 51.59 & 4.12 & 51.31 \\
\cline{1-6}
\multirow[t]{4}{*}{2nd Derivative} & No Reranker & 1.78 & 12.42 & 1.77 & 12.61 \\
 & GTE ML Reranker Base & 2.46 & 27.72 & 2.61 & 53.69 \\
 & Jina Reranker v1 Tiny EN & 4.49 & 161.05 & 3.09 & 99.01 \\
 & MXBAI Rerank Base v2 & 8.45 & 217.15 & 6.79 & 188.22 \\
\cline{1-6}
\bottomrule
\end{tabular}
\end{table*}


\begin{figure}[htbp]
  \centering
  \makebox[\textwidth][c]{%
    \begin{minipage}{0.90\paperwidth}
      \centering
      % first row: F1 & precision
      \begin{minipage}[b]{0.49\textwidth}
        \centering
        \includegraphics[width=\textwidth]{reranker/avg_f1_score_bar_faceted.png}
        \\[1ex]\footnotesize (a) Average F1 score by facet
      \end{minipage}
      \hfill
      \begin{minipage}[b]{0.49\textwidth}
        \centering
        \includegraphics[width=\textwidth]{reranker/avg_precision_bar_faceted.png}
        \\[1ex]\footnotesize (b) Average precision by facet
      \end{minipage}

      \vspace{1ex}

      % second row: recall
      \begin{minipage}[b]{0.49\textwidth}
        \centering
        \includegraphics[width=\textwidth]{reranker/avg_recall_bar_faceted.png}
        \\[1ex]\footnotesize (c) Average recall by facet
      \end{minipage}

      \vspace{1ex}

      % third row: items & retrieved docs
      \begin{minipage}[b]{0.49\textwidth}
        \centering
        \includegraphics[width=\textwidth]{reranker/avg_num_items_passed_bar_faceted.png}
        \\[1ex]\footnotesize (d) Average number of items passed by facet
      \end{minipage}
      \hfill
      \begin{minipage}[b]{0.49\textwidth}
        \centering
        \includegraphics[width=\textwidth]{reranker/avg_num_retrieved_docs_passed_bar_faceted.png}
        \\[1ex]\footnotesize (e) Average number of retrieved documents passed by facet
      \end{minipage}
    \end{minipage}% 
  }
  \caption{\footnotesize Comparison of average metrics by model facet:
    (a) F1 score, (b) precision, (c) recall, (d) number of items passed,
    and (e) number of retrieved documents passed.}
  \label{fig:avg_metrics_faceted}
\end{figure}

\section{Experiment 3: Prompt Engineering to Prevent Hallucination}
\label{sec:exp_generator_prompt}
This experiment focuses on the generation part of the RAG pipeline, evaluating a wide range of Large Language Models to serve as the generator. The primary goal was to find the best-performing model and the optimal prompt and temperature configuration for our specific use case.

\subsection{Motivation and Baseline}
The evaluation started with a baseline configuration of \textbf{GPT-4o} using prompt \texttt{P1} and a temperature of \texttt{0.2}. As new and more powerful models were released during the course of this research, they were systematically evaluated against this baseline. This continuous evaluation process allowed us to stay at the cutting edge and select the most suitable model for integration into a production RAG system, ensuring that any replacement would offer superior performance.

\subsection{Evaluation Methodology}
We used the \textbf{DeepEval} framework \autocite{deepeval2023}, which employs an LLM-as-a-Judge approach to score the generator's output. For subjective, use-case-specific evaluations, we utilized DeepEval's \textbf{G-Eval} functionality, which is inspired by the G-Eval framework \autocite{liu2023geval} and uses a chain-of-thought process to evaluate outputs against custom criteria. We defined two such metrics:
\begin{itemize}
    \item \textbf{Correctness (C):} This metric determines whether the actual output is factually correct based on the expected output. It heavily penalizes the omission of key details and the inclusion of information that contradicts the expected output.
    \item \textbf{Specific Information Accuracy (SIA):} This metric evaluates whether the model's response appropriately uses information from the context without introducing specific details (names, places, numbers) that are not explicitly provided. It is particularly important for testing the model's ability to recognize when a question cannot be answered from the given context. For example, a high SIA score is awarded if the model correctly states it cannot answer a question like "Who is Cinderella?" when given an anonymized version of the story, as this demonstrates it is not relying on its own parametric knowledge.
\end{itemize}

In addition to our custom G-Eval metrics, we used several of DeepEval's standard metrics to assess other critical qualities of the generated output:
\begin{itemize}
    \item \textbf{Answer Relevancy (AR):} This metric measures how relevant the generated output is to the user's input query. It ensures that the model's response is on-topic and directly addresses the question asked \autocite{deepeval2023}.
    \item \textbf{Faithfulness (F):} This metric evaluates whether the generated output factually aligns with the information present in the retrieved context. A high score indicates that the model did not invent facts and stayed true to the source material \autocite{deepeval2023}.
    \item \textbf{Hallucination (H):} This metric determines if the model generates factually incorrect information by comparing the output to the provided context. It is a crucial measure for ensuring the trustworthiness of the RAG system \autocite{deepeval2023}.
\end{itemize}


\subsection{Results and Discussion}
The comprehensive results are presented in Table \ref{tab:deepeval_results}. The baseline configuration (GPT-4o, P1, Temp 0.2) achieved a total score of 370.52. The results show that several newer models and prompt configurations were able to significantly outperform this baseline.

For instance, the \textbf{O4 Mini} model using prompt \texttt{P2} and a temperature of \texttt{0.2} achieved the highest total score of \textbf{403.86}. This demonstrates the value of continuous evaluation, as it allowed us to identify a model that not only performs better than the original baseline but also to determine the ideal prompt (\texttt{P2}) and temperature (\texttt{0.2}) for it. These findings are critical for deploying a RAG system that is not only accurate and faithful but also robust in handling queries that cannot be answered from the available context.

\input{chapter7_tables_prompt_engineering.tex}

\subsection{Comparative Analysis Against GPT-4o}
\label{sec:comparative_analysis_gpt4o}

To better understand the relative performance of each model, we calculated the delta between each model's total score and the baseline score of GPT-4o (370.52). The results are summarized in Table \ref{tab:deepeval_delta_results}.

\begin{table}[!htbp]
\centering
\small
\caption{Total Score Delta vs. GPT-4o (Highest Scores)}
\label{tab:deepeval_delta_results}
\begin{tabular}{|l|c|c|c|}
\hline
\textbf{Model} & \textbf{Prompt} & \textbf{Temp} & \textbf{Delta from GPT-4o} \\
\hline
O4 Mini & P2 & 0.2 & +33.34 \\
O1 & P1 & 0.0 & +33.33 \\
Gemini 2.5 Pro & P2 & 0.2 & +32.05 \\
O3 & P2 & 0.2 & +30.75 \\
Claude 3.5 Sonnet v2 & P2 & 0.2 & +29.47 \\
GPT-4.1 & P1 & 0.2 & +29.46 \\
Claude 3.7 Sonnet & P2 & 0.2 & +19.23 \\
Gemini 1.5 Flash & P2 & 0.2 & +19.23 \\
Gemini 2.5 Flash & P2 & 0.2 & +19.23 \\
Claude 4.0 Sonnet & P2 & 0.2 & +17.96 \\
O3 Mini & P2 & 0.2 & +11.54 \\
Claude 3.5 Sonnet & P1 & 0.0 & +3.83 \\
GPT-4 Omni Mini & P2 & 0.2 & +1.27 \\
GPT-4 Omni & P2 & 0.2 & +1.27 \\
\hline
\multicolumn{4}{|c|}{\textbf{GPT-4o Baseline Total Score: 370.52}} \\
\hline
GPT-4.1 Nano & P1 & 0.2 & -2.57 \\
Gemini 2.0 Flash & P2 & 0.0 & -6.43 \\
Claude 3 Haiku & P2 & 0.0 & -9.00 \\
Gemini 1.5 Pro & P2 & 0.2 & -9.00 \\
\hline
\end{tabular}
\end{table}

\subsubsection{Inference from the Results}
The delta analysis reveals several key insights. Firstly, the top-performing models, such as \textbf{O4 Mini}, \textbf{O1}, and \textbf{Gemini 2.5 Pro}, show a significant improvement over the GPT-4o baseline, with score increases exceeding 30 points. This underscores the rapid advancements in generative models, where newer architectures can yield substantial performance gains.

Secondly, the choice of prompt and temperature settings is evidently crucial. For many models, there is a considerable performance variance between different prompt versions (P1 vs. P2) and temperature settings. For instance, the O1 model with prompt P1 scores much higher than with P2 (403.85 vs 396.16: +7.69 points). This highlights the necessity of meticulous prompt engineering and parameter tuning to unlock a model's full potential.
