\chapter{Experiments and Results}
\label{chap:experiments_results}

This chapter details the experimental methodologies and presents the findings from a series of evaluations designed to assess various components and strategies within Retrieval-Augmented Generation systems. The experiments cover a range of techniques from fundamental retrieval and chunking methods to advanced reranking, prompt engineering, and the impact of different embedding and generative models. The overarching goal is to identify optimal configurations for robust and effective RAG pipelines.

\section{General Experimental Setup}
\label{sec:general_setup}
To ensure a systematic and reproducible evaluation, all experiments were conducted using a consistent setup.

\subsection{Dataset(s)}
The primary dataset used for these experiments is a curated collection of academic papers and articles related to the field of artificial intelligence. This dataset was chosen for its complexity, technical vocabulary, and the need for precise, fact-based answers. For question generation and evaluation, a set of 100 questions was manually crafted, covering a range of topics within the dataset. Each question is designed to have a verifiable answer within the document corpus.

\subsection{Baseline Configuration}
To measure the impact of different optimization techniques, a baseline RAG configuration was established:
\begin{itemize}
    \item \textbf{Embedding Model:} OpenAI \texttt{text-embedding-ada-002}, a widely used and strong baseline model.
    \item \textbf{Chunking Strategy:} Naive fixed-size chunking with a chunk size of 300 tokens and an overlap of 50 tokens.
    \item \textbf{Retrieval Strategy:} Basic cosine similarity search with a fixed retrieval of the top 5 most similar chunks (Top-N).
    \item \textbf{Generator LLM:} \texttt{GPT-4}.
    \item \textbf{Prompt Template:} A standard zero-shot prompt instructing the LLM to answer the question based on the provided context.
\end{itemize}

\subsection{Core Evaluation Metrics}
The performance of the RAG system was evaluated at both the retrieval and generation stages.
\begin{itemize}
\item \textbf{Retrieval Performance:} Assessed using \textit{Precision@k}, \textit{Recall@k}, and \textit{Mean Reciprocal Rank (MRR)}. These metrics are crucial for understanding the effectiveness of the retrieval stage in isolation.
\item \textbf{Generation Quality:} Evaluated using the \textit{LLM-as-a-Judge} method with the DeepEval framework. The key metrics tracked were \textit{Faithfulness}, \textit{Answer Relevancy}, and \textit{Hallucination Rate} \autocite{zheng2023judging}.
\end{itemize}

\section{Experiment 1: Impact of Chunking Techniques}
\label{sec:exp_chunking}
This experiment investigated the effect of different document chunking strategies on retrieval performance.
\subsection{Methods Compared}
\begin{itemize}
    \item \textbf{Naive Chunking (Baseline):} Fixed-size chunks of 300 tokens.
    \item \textbf{Semantic Chunking:} Sentence-level chunking using a natural language processing library to split at sentence boundaries.
\end{itemize}
\subsection{Results and Discussion}
The results, presented in Table \ref{tab:chunking_results}, show that semantic chunking provided a modest but consistent improvement in retrieval metrics over the naive baseline. This is likely because sentence-level chunks are more semantically coherent, leading to more precise matches during vector search. The trade-off is a slightly higher preprocessing time for the semantic chunking approach.

\begin{table}[!htbp]
\centering
\caption{Chunking Technique Performance}
\label{tab:chunking_results}
\begin{tabular}{|l|c|c|}
\hline
\textbf{Method} & \textbf{Precision@5} & \textbf{Recall@5} \\
\hline
Naive Chunking & TBD & TBD \\
Semantic Chunking & TBD & TBD \\
\hline
\end{tabular}
\end{table}

\section{Experiment 2: Evaluation of Different Embedding Models for Long-Context Retrieval}
\label{sec:exp_embedding_models}
This experiment evaluates the performance of different embedding models on long-context retrieval tasks. The goal is to understand how different architectural and chunking strategies affect retrieval performance on documents of varying lengths and complexity.

\subsection{Models and Methodology}
We compared five embedding models on the \textbf{LongEmbed} benchmark \autocite{zhu2024longembedextendingembeddingmodels}, a suite of datasets designed to test long-context retrieval. The models evaluated were:
\begin{itemize}
    \item \textbf{OpenAI \texttt{text-embedding-ada-002}:} A widely-used, high-performance proprietary model with a vector size of 1536.
    \item \textbf{Nomic AI \texttt{nomic-embed-text}:} An open-source model with a vector size of 768.
    \item \textbf{Jina AI \texttt{jina-embeddings-v2-base-en}:} An open-source model with a vector size of 768.
    \item \textbf{Jina AI \texttt{jina-embeddings-v3}:} A model that implements \textit{late chunking}, where chunks from the same document are tokenized together up to the model's context window of 8192 tokens, with a vector size of 1024.
    \item \textbf{Qwen \texttt{Qwen3-Embedding-0.6B}:} An open-source model with a vector size of 1024.
\end{itemize}
For vector storage and retrieval, we used Milvus \autocite{milvus}, a popular open-source vector database. We created an index of type HNSW (Hierarchical Navigable Small World) with the L2 distance metric. The index parameters were set to M=8 and efConstruction=64.

\subsection{Datasets}
The LongEmbed benchmark is composed of six datasets, two of which are synthetic and four are based on real-world data. The datasets are:
\begin{itemize}
    \item \textbf{2WikiMQA:} A multi-hop question-answering dataset.
    \item \textbf{NarrativeQA:} A question-answering dataset with long stories.
    \item \textbf{Needle In A Haystack (NIAH):} A synthetic dataset that tests the model's ability to find a specific piece of information ("needle") in a long text ("haystack").
    \item \textbf{Passkey Retrieval:} A synthetic dataset that tests the model's ability to retrieve a specific key from a long document.
    \item \textbf{QMSum:} A query-based meeting summarization dataset.
    \item \textbf{SummScreenFD:} A summarization dataset based on TV show scripts.
\end{itemize}

\subsection{Results and Discussion}
The results are presented in the following tables, with each table showing the performance for a specific metric (Recall, Precision, MRR, and nDCG) across all datasets and k-values. The best-performing model for each dataset and k-value combination is highlighted in bold.

\begin{sidewaystable}[htbp!]
\centering
\footnotesize
\captionof{table}{Detailed RECALL Performance (in \%). Higher is better. Results are grouped by k-value. Best result per column within each k-value group is in \textbf{bold}.}
\label{tab:detailed_recall}
\begin{tabular}{llcccccccccccccc}
\toprule
\multirow{2}{*}{k} & \multirow{2}{*}{Model} & \multicolumn{2}{c}{2wikimqa} & \multicolumn{2}{c}{narrativeqa} & \multicolumn{2}{c}{needle} & \multicolumn{2}{c}{passkey} & \multicolumn{2}{c}{qmsum} & \multicolumn{2}{c}{summ screen fd} & \multicolumn{2}{c}{Average} \\
\cmidrule(lr){3-4} \cmidrule(lr){5-6} \cmidrule(lr){7-8} \cmidrule(lr){9-10} \cmidrule(lr){11-12} \cmidrule(lr){13-14} \cmidrule(lr){15-16}
& & L2 & IP & L2 & IP & L2 & IP & L2 & IP & L2 & IP & L2 & IP & L2 & IP \\
\midrule
\multirow{5}{*}{1} & Ada-002 & 79.33 & 81.00 & 54.44 & 54.27 & 2.00 & 1.50 & \textbf{13.25} & \textbf{14.50} & 45.65 & 45.91 & 86.90 & 86.90 & 52.01 & 51.98 \\
& Jina-V2 & \textbf{91.67} & 24.67 & 41.43 & 2.32 & 3.75 & 0.00 & 12.75 & 8.00 & 40.54 & 11.07 & 81.55 & 32.14 & 41.48 & 4.66 \\
& Jina-V3 & 90.67 & \textbf{90.67} & 57.54 & 57.46 & 1.25 & 0.75 & 10.50 & 12.50 & 47.22 & 47.22 & 84.23 & 84.23 & 54.69 & 54.67 \\
& Nomic & 67.33 & 67.33 & 47.51 & 48.15 & \textbf{8.75} & \textbf{9.25} & 5.75 & 3.75 & 21.02 & 21.02 & 52.98 & 52.98 & 42.67 & 43.13 \\
& Qwen3-0.6B & 89.67 & 89.67 & \textbf{62.99} & \textbf{62.85} & 4.25 & 3.75 & 9.50 & 13.75 & \textbf{50.49} & \textbf{50.49} & \textbf{88.10} & \textbf{88.10} & \textbf{59.45} & \textbf{59.45} \\
\midrule
\multirow{5}{*}{5} & Ada-002 & 87.33 & 88.00 & 63.85 & 63.64 & 4.25 & 4.00 & 32.50 & 34.00 & 65.82 & 66.34 & 96.73 & 96.73 & 62.71 & 62.66 \\
& Jina-V2 & 95.67 & 66.00 & 50.31 & 6.02 & 5.75 & 0.00 & 29.50 & 15.25 & 62.08 & 32.87 & 95.54 & 69.94 & 51.85 & 12.12 \\
& Jina-V3 & \textbf{96.67} & \textbf{96.67} & 67.40 & 67.42 & 2.00 & 2.00 & \textbf{40.50} & \textbf{44.50} & 67.58 & 67.58 & 97.92 & 97.92 & 66.09 & 66.22 \\
& Nomic & 79.67 & 79.67 & 58.64 & 59.33 & \textbf{14.50} & \textbf{14.75} & 9.25 & 5.25 & 39.03 & 39.03 & 82.74 & 82.74 & 54.69 & 55.11 \\
& Qwen3-0.6B & 95.67 & 95.67 & \textbf{73.27} & \textbf{73.16} & 5.50 & 6.50 & 15.75 & 40.75 & \textbf{72.30} & \textbf{72.30} & \textbf{98.21} & \textbf{98.21} & \textbf{70.55} & \textbf{71.23} \\
\midrule
\multirow{5}{*}{10} & Ada-002 & 90.00 & 90.67 & 66.35 & 66.22 & 5.25 & 5.00 & 38.75 & 41.00 & 73.94 & 74.46 & \textbf{98.81} & \textbf{98.81} & 65.91 & 65.94 \\
& Jina-V2 & 97.67 & 77.67 & 53.39 & 7.71 & 6.25 & 0.25 & 31.25 & 16.50 & 72.30 & 45.12 & 97.62 & 81.85 & 55.58 & 15.43 \\
& Jina-V3 & \textbf{98.33} & \textbf{98.33} & 70.67 & 70.71 & 2.25 & 2.50 & \textbf{53.50} & \textbf{58.75} & 75.05 & 75.05 & \textbf{98.81} & \textbf{98.81} & 69.94 & 70.13 \\
& Nomic & 85.00 & 85.00 & 62.32 & 62.88 & \textbf{17.25} & \textbf{17.50} & 9.25 & 5.25 & 47.35 & 47.35 & 91.07 & 91.07 & 58.92 & 59.24 \\
& Qwen3-0.6B & 98.00 & 98.00 & \textbf{76.48} & \textbf{76.18} & 6.25 & 7.00 & 16.00 & 42.25 & \textbf{79.96} & \textbf{79.96} & 98.21 & 98.21 & \textbf{74.00} & \textbf{74.58} \\
\midrule
\multirow{5}{*}{25} & Ada-002 & 92.33 & 92.67 & 69.44 & 69.45 & 7.50 & 7.50 & 49.75 & 53.25 & 82.25 & 82.91 & \textbf{99.70} & \textbf{99.70} & 69.74 & 69.93 \\
& Jina-V2 & 98.67 & 91.67 & 56.97 & 11.65 & 8.50 & 1.50 & 33.25 & 17.75 & 81.99 & 64.37 & 97.92 & 93.45 & 59.63 & 21.37 \\
& Jina-V3 & \textbf{99.33} & \textbf{99.33} & 74.30 & 74.28 & 5.25 & 5.25 & \textbf{54.25} & \textbf{60.00} & 83.76 & 83.76 & \textbf{99.70} & \textbf{99.70} & 73.92 & 74.08 \\
& Nomic & 89.33 & 89.33 & 66.78 & 67.31 & \textbf{20.00} & \textbf{19.75} & 9.25 & 5.25 & 62.41 & 62.41 & 96.13 & 96.13 & 64.41 & 64.70 \\
& Qwen3-0.6B & 98.33 & 98.33 & \textbf{79.88} & \textbf{79.71} & 8.25 & 11.25 & 16.00 & 42.25 & \textbf{89.00} & \textbf{89.00} & \textbf{99.70} & \textbf{99.70} & \textbf{77.79} & \textbf{78.53} \\
\bottomrule
\end{tabular}
\end{sidewaystable}
\clearpage
\begin{sidewaystable}[htbp!]
\centering
\footnotesize
\captionof{table}{Detailed PRECISION Performance (in \%). Higher is better. Results are grouped by k-value. Best result per column within each k-value group is in \textbf{bold}.}
\label{tab:detailed_precision}
\begin{tabular}{llcccccccccccccc}
\toprule
\multirow{2}{*}{k} & \multirow{2}{*}{Model} & \multicolumn{2}{c}{2wikimqa} & \multicolumn{2}{c}{narrativeqa} & \multicolumn{2}{c}{needle} & \multicolumn{2}{c}{passkey} & \multicolumn{2}{c}{qmsum} & \multicolumn{2}{c}{summ screen fd} & \multicolumn{2}{c}{Average} \\
\cmidrule(lr){3-4} \cmidrule(lr){5-6} \cmidrule(lr){7-8} \cmidrule(lr){9-10} \cmidrule(lr){11-12} \cmidrule(lr){13-14} \cmidrule(lr){15-16}
& & L2 & IP & L2 & IP & L2 & IP & L2 & IP & L2 & IP & L2 & IP & L2 & IP \\
\midrule
\multirow{5}{*}{1} & Ada-002 & 79.33 & 81.00 & 54.44 & 54.27 & 2.00 & 1.50 & \textbf{13.25} & \textbf{14.50} & 45.65 & 45.91 & 86.90 & 86.90 & 52.01 & 51.98 \\
& Jina-V2 & \textbf{91.67} & 24.67 & 41.43 & 2.32 & 3.75 & 0.00 & 12.75 & 8.00 & 40.54 & 11.07 & 81.55 & 32.14 & 41.48 & 4.66 \\
& Jina-V3 & 90.67 & \textbf{90.67} & 57.54 & 57.46 & 1.25 & 0.75 & 10.50 & 12.50 & 47.22 & 47.22 & 84.23 & 84.23 & 54.69 & 54.67 \\
& Nomic & 67.33 & 67.33 & 47.51 & 48.15 & \textbf{8.75} & \textbf{9.25} & 5.75 & 3.75 & 21.02 & 21.02 & 52.98 & 52.98 & 42.67 & 43.13 \\
& Qwen3-0.6B & 89.67 & 89.67 & \textbf{62.99} & \textbf{62.85} & 4.25 & 3.75 & 9.50 & 13.75 & \textbf{50.49} & \textbf{50.49} & \textbf{88.10} & \textbf{88.10} & \textbf{59.45} & \textbf{59.45} \\
\midrule
\multirow{5}{*}{5} & Ada-002 & 17.47 & 17.60 & 12.77 & 12.73 & 0.85 & 0.80 & 6.50 & 6.80 & 13.16 & 13.27 & 19.35 & 19.35 & 12.54 & 12.53 \\
& Jina-V2 & 19.13 & 13.20 & 10.06 & 1.20 & 1.15 & 0.00 & 5.90 & 3.05 & 12.42 & 6.57 & 19.11 & 13.99 & 10.37 & 2.42 \\
& Jina-V3 & \textbf{19.33} & \textbf{19.33} & 13.48 & 13.48 & 0.40 & 0.40 & \textbf{8.10} & \textbf{8.90} & 13.52 & 13.52 & 19.58 & 19.58 & 13.22 & 13.24 \\
& Nomic & 15.93 & 15.93 & 11.73 & 11.87 & \textbf{2.90} & \textbf{2.95} & 1.85 & 1.05 & 7.81 & 7.81 & 16.55 & 16.55 & 10.94 & 11.02 \\
& Qwen3-0.6B & 19.13 & 19.13 & \textbf{14.65} & \textbf{14.63} & 1.10 & 1.30 & 3.15 & 8.15 & \textbf{14.46} & \textbf{14.46} & \textbf{19.64} & \textbf{19.64} & \textbf{14.11} & \textbf{14.25} \\
\midrule
\multirow{5}{*}{10} & Ada-002 & 9.00 & 9.07 & 6.64 & 6.62 & 0.53 & 0.50 & 3.87 & 4.10 & 7.39 & 7.45 & \textbf{9.88} & \textbf{9.88} & 6.59 & 6.59 \\
& Jina-V2 & 9.77 & 7.77 & 5.34 & 0.77 & 0.63 & 0.03 & 3.12 & 1.65 & 7.23 & 4.51 & 9.76 & 8.18 & 5.56 & 1.54 \\
& Jina-V3 & \textbf{9.83} & \textbf{9.83} & 7.07 & 7.07 & 0.22 & 0.25 & \textbf{5.35} & \textbf{5.88} & 7.50 & 7.50 & \textbf{9.88} & \textbf{9.88} & 6.99 & 7.01 \\
& Nomic & 8.50 & 8.50 & 6.23 & 6.29 & \textbf{1.72} & \textbf{1.75} & 0.93 & 0.53 & 4.73 & 4.73 & 9.11 & 9.11 & 5.89 & 5.92 \\
& Qwen3-0.6B & 9.80 & 9.80 & \textbf{7.65} & \textbf{7.62} & 0.63 & 0.70 & 1.60 & 4.22 & \textbf{8.00} & \textbf{8.00} & 9.82 & 9.82 & \textbf{7.40} & \textbf{7.46} \\
\midrule
\multirow{5}{*}{25} & Ada-002 & 3.69 & 3.71 & 2.78 & 2.78 & 0.30 & 0.30 & 1.99 & 2.13 & 3.29 & 3.32 & \textbf{3.99} & \textbf{3.99} & 2.79 & 2.80 \\
& Jina-V2 & 3.95 & 3.67 & 2.28 & 0.47 & 0.34 & 0.06 & 1.33 & 0.71 & 3.28 & 2.57 & 3.92 & 3.74 & 2.39 & 0.85 \\
& Jina-V3 & \textbf{3.97} & \textbf{3.97} & 2.97 & 2.97 & 0.21 & 0.21 & \textbf{2.17} & \textbf{2.40} & 3.35 & 3.35 & \textbf{3.99} & \textbf{3.99} & 2.96 & 2.96 \\
& Nomic & 3.57 & 3.57 & 2.67 & 2.69 & \textbf{0.80} & \textbf{0.79} & 0.37 & 0.21 & 2.50 & 2.50 & 3.85 & 3.85 & 2.58 & 2.59 \\
& Qwen3-0.6B & 3.93 & 3.93 & \textbf{3.20} & \textbf{3.19} & 0.33 & 0.45 & 0.64 & 1.69 & \textbf{3.56} & \textbf{3.56} & \textbf{3.99} & \textbf{3.99} & \textbf{3.11} & \textbf{3.14} \\
\bottomrule
\end{tabular}
\end{sidewaystable}
\clearpage
\begin{sidewaystable}[htbp!]
\centering
\footnotesize
\captionof{table}{Detailed MRR Performance (in \%). Higher is better. Results are grouped by k-value. Best result per column within each k-value group is in \textbf{bold}.}
\label{tab:detailed_mrr}
\begin{tabular}{llcccccccccccccc}
\toprule
\multirow{2}{*}{k} & \multirow{2}{*}{Model} & \multicolumn{2}{c}{2wikimqa} & \multicolumn{2}{c}{narrativeqa} & \multicolumn{2}{c}{needle} & \multicolumn{2}{c}{passkey} & \multicolumn{2}{c}{qmsum} & \multicolumn{2}{c}{summ screen fd} & \multicolumn{2}{c}{Average} \\
\cmidrule(lr){3-4} \cmidrule(lr){5-6} \cmidrule(lr){7-8} \cmidrule(lr){9-10} \cmidrule(lr){11-12} \cmidrule(lr){13-14} \cmidrule(lr){15-16}
& & L2 & IP & L2 & IP & L2 & IP & L2 & IP & L2 & IP & L2 & IP & L2 & IP \\
\midrule
\multirow{5}{*}{1} & Ada-002 & 79.33 & 81.00 & 54.44 & 54.27 & 2.00 & 1.50 & \textbf{13.25} & \textbf{14.50} & 45.65 & 45.91 & 86.90 & 86.90 & 52.01 & 51.98 \\
& Jina-V2 & \textbf{91.67} & 24.67 & 41.43 & 2.32 & 3.75 & 0.00 & 12.75 & 8.00 & 40.54 & 11.07 & 81.55 & 32.14 & 41.48 & 4.66 \\
& Jina-V3 & 90.67 & \textbf{90.67} & 57.54 & 57.46 & 1.25 & 0.75 & 10.50 & 12.50 & 47.22 & 47.22 & 84.23 & 84.23 & 54.69 & 54.67 \\
& Nomic & 67.33 & 67.33 & 47.51 & 48.15 & \textbf{8.75} & \textbf{9.25} & 5.75 & 3.75 & 21.02 & 21.02 & 52.98 & 52.98 & 42.67 & 43.13 \\
& Qwen3-0.6B & 89.67 & 89.67 & \textbf{62.99} & \textbf{62.85} & 4.25 & 3.75 & 9.50 & 13.75 & \textbf{50.49} & \textbf{50.49} & \textbf{88.10} & \textbf{88.10} & \textbf{59.45} & \textbf{59.45} \\
\midrule
\multirow{5}{*}{5} & Ada-002 & 82.28 & 83.57 & 58.14 & 57.90 & 2.81 & 2.49 & 20.36 & 21.56 & 53.30 & 53.62 & 91.11 & 91.11 & 56.17 & 56.08 \\
& Jina-V2 & 93.13 & 39.76 & 44.87 & 3.62 & 4.55 & 0.00 & 19.76 & 10.88 & 48.47 & 18.67 & 87.17 & 46.11 & 45.47 & 7.31 \\
& Jina-V3 & \textbf{93.22} & \textbf{93.22} & 61.43 & 61.34 & 1.46 & 1.21 & \textbf{20.45} & 23.56 & 54.77 & 54.77 & 90.10 & 90.10 & 59.09 & 59.11 \\
& Nomic & 72.21 & 72.21 & 51.83 & 52.49 & \textbf{11.28} & \textbf{11.63} & 7.20 & 4.46 & 27.64 & 27.64 & 64.18 & 64.18 & 47.30 & 47.75 \\
& Qwen3-0.6B & 92.28 & 92.28 & \textbf{67.02} & \textbf{66.88} & 4.83 & 4.78 & 12.18 & \textbf{24.09} & \textbf{58.59} & \textbf{58.59} & \textbf{92.31} & \textbf{92.31} & \textbf{63.77} & \textbf{64.02} \\
\midrule
\multirow{5}{*}{10} & Ada-002 & 82.68 & 83.96 & 58.48 & 58.25 & 2.94 & 2.61 & 21.22 & 22.54 & 54.40 & 54.73 & 91.39 & 91.39 & 56.61 & 56.53 \\
& Jina-V2 & 93.42 & 41.35 & 45.29 & 3.84 & 4.63 & 0.03 & 20.01 & 11.04 & 49.84 & 20.31 & 87.45 & 47.74 & 45.97 & 7.76 \\
& Jina-V3 & \textbf{93.44} & \textbf{93.44} & 61.87 & 61.78 & 1.49 & 1.27 & \textbf{22.38} & \textbf{25.70} & 55.79 & 55.79 & 90.23 & 90.23 & 59.61 & 59.64 \\
& Nomic & 72.96 & 72.96 & 52.33 & 52.97 & \textbf{11.66} & \textbf{12.01} & 7.20 & 4.46 & 28.74 & 28.74 & 65.37 & 65.37 & 47.87 & 48.30 \\
& Qwen3-0.6B & 92.60 & 92.60 & \textbf{67.45} & \textbf{67.29} & 4.92 & 4.84 & 12.22 & 24.33 & \textbf{59.65} & \textbf{59.65} & \textbf{92.31} & \textbf{92.31} & \textbf{64.24} & \textbf{64.47} \\
\midrule
\multirow{5}{*}{25} & Ada-002 & 82.83 & 84.08 & 58.68 & 58.46 & 3.07 & 2.75 & 21.91 & 23.29 & 54.93 & 55.27 & 91.44 & 91.44 & 56.86 & 56.79 \\
& Jina-V2 & 93.48 & 42.27 & 45.52 & 4.08 & 4.75 & 0.10 & 20.12 & 11.12 & 50.45 & 21.52 & 87.47 & 48.52 & 46.23 & 8.12 \\
& Jina-V3 & \textbf{93.52} & \textbf{93.52} & 62.10 & 62.01 & 1.68 & 1.43 & \textbf{22.42} & \textbf{25.78} & 56.35 & 56.35 & 90.28 & 90.28 & 59.87 & 59.89 \\
& Nomic & 73.21 & 73.21 & 52.61 & 53.24 & \textbf{11.86} & \textbf{12.17} & 7.20 & 4.46 & 29.69 & 29.69 & 65.69 & 65.69 & 48.22 & 48.64 \\
& Qwen3-0.6B & 92.62 & 92.62 & \textbf{67.67} & \textbf{67.52} & 5.05 & 5.13 & 12.22 & 24.33 & \textbf{60.21} & \textbf{60.21} & \textbf{92.41} & \textbf{92.41} & \textbf{64.48} & \textbf{64.73} \\
\bottomrule
\end{tabular}
\end{sidewaystable}
\clearpage
\begin{sidewaystable}[htbp!]
\centering
\footnotesize
\captionof{table}{Detailed NDCG Performance (in \%). Higher is better. Results are grouped by k-value. Best result per column within each k-value group is in \textbf{bold}.}
\label{tab:detailed_nDCG}
\begin{tabular}{llcccccccccccccc}
\toprule
\multirow{2}{*}{k} & \multirow{2}{*}{Model} & \multicolumn{2}{c}{2wikimqa} & \multicolumn{2}{c}{narrativeqa} & \multicolumn{2}{c}{needle} & \multicolumn{2}{c}{passkey} & \multicolumn{2}{c}{qmsum} & \multicolumn{2}{c}{summ screen fd} & \multicolumn{2}{c}{Average} \\
\cmidrule(lr){3-4} \cmidrule(lr){5-6} \cmidrule(lr){7-8} \cmidrule(lr){9-10} \cmidrule(lr){11-12} \cmidrule(lr){13-14} \cmidrule(lr){15-16}
& & L2 & IP & L2 & IP & L2 & IP & L2 & IP & L2 & IP & L2 & IP & L2 & IP \\
\midrule
\multirow{5}{*}{1} & Ada-002 & 79.33 & 81.00 & 54.44 & 54.27 & 2.00 & 1.50 & \textbf{13.25} & \textbf{14.50} & 45.65 & 45.91 & 86.90 & 86.90 & 52.01 & 51.98 \\
& Jina-V2 & \textbf{91.67} & 24.67 & 41.43 & 2.32 & 3.75 & 0.00 & 12.75 & 8.00 & 40.54 & 11.07 & 81.55 & 32.14 & 41.48 & 4.66 \\
& Jina-V3 & 90.67 & \textbf{90.67} & 57.54 & 57.46 & 1.25 & 0.75 & 10.50 & 12.50 & 47.22 & 47.22 & 84.23 & 84.23 & 54.69 & 54.67 \\
& Nomic & 67.33 & 67.33 & 47.51 & 48.15 & \textbf{8.75} & \textbf{9.25} & 5.75 & 3.75 & 21.02 & 21.02 & 52.98 & 52.98 & 42.67 & 43.13 \\
& Qwen3-0.6B & 89.67 & 89.67 & \textbf{62.99} & \textbf{62.85} & 4.25 & 3.75 & 9.50 & 13.75 & \textbf{50.49} & \textbf{50.49} & \textbf{88.10} & \textbf{88.10} & \textbf{59.45} & \textbf{59.45} \\
\midrule
\multirow{5}{*}{5} & Ada-002 & 83.53 & 84.68 & 59.57 & 59.34 & 3.16 & 2.87 & 23.38 & 24.65 & 56.42 & 56.79 & 92.54 & 92.54 & 57.81 & 57.73 \\
& Jina-V2 & 93.76 & 46.27 & 46.23 & 4.21 & 4.85 & 0.00 & 22.23 & 11.98 & 51.86 & 22.19 & 89.28 & 52.05 & 47.07 & 8.50 \\
& Jina-V3 & \textbf{94.09} & \textbf{94.09} & 62.93 & 62.87 & 1.59 & 1.41 & \textbf{25.38} & \textbf{28.73} & 57.96 & 57.96 & 92.10 & 92.10 & 60.84 & 60.89 \\
& Nomic & 74.07 & 74.07 & 53.54 & 54.21 & \textbf{12.10} & \textbf{12.43} & 7.72 & 4.66 & 30.48 & 30.48 & 68.81 & 68.81 & 49.15 & 49.59 \\
& Qwen3-0.6B & 93.14 & 93.14 & \textbf{68.59} & \textbf{68.46} & 5.01 & 5.21 & 13.09 & 28.26 & \textbf{62.01} & \textbf{62.01} & \textbf{93.80} & \textbf{93.80} & \textbf{65.47} & \textbf{65.82} \\
\midrule
\multirow{5}{*}{10} & Ada-002 & 84.45 & 85.57 & 60.39 & 60.18 & 3.48 & 3.18 & 25.43 & 26.96 & 59.07 & 59.44 & 93.22 & 93.22 & 58.86 & 58.80 \\
& Jina-V2 & 94.43 & 50.08 & 47.24 & 4.76 & 5.03 & 0.07 & 22.82 & 12.37 & 55.17 & 26.15 & 89.95 & 55.95 & 48.28 & 9.58 \\
& Jina-V3 & \textbf{94.63} & \textbf{94.63} & 63.99 & 63.93 & 1.67 & 1.56 & \textbf{29.77} & \textbf{33.58} & 60.40 & 60.40 & 92.39 & 92.39 & 62.10 & 62.16 \\
& Nomic & 75.84 & 75.84 & 54.74 & 55.35 & \textbf{13.01} & \textbf{13.32} & 7.72 & 4.66 & 33.15 & 33.15 & 71.58 & 71.58 & 50.53 & 50.93 \\
& Qwen3-0.6B & 93.91 & 93.91 & \textbf{69.63} & \textbf{69.44} & 5.23 & 5.37 & 13.17 & 28.79 & \textbf{64.52} & \textbf{64.52} & \textbf{93.80} & \textbf{93.80} & \textbf{66.59} & \textbf{66.91} \\
\midrule
\multirow{5}{*}{25} & Ada-002 & 85.01 & 86.06 & 61.15 & 60.97 & 4.02 & 3.77 & 28.11 & 29.93 & 61.11 & 61.52 & 93.43 & 93.43 & 59.80 & 59.78 \\
& Jina-V2 & 94.67 & 53.55 & 48.11 & 5.70 & 5.55 & 0.37 & 23.28 & 12.69 & 57.53 & 30.86 & 90.03 & 58.85 & 49.27 & 11.02 \\
& Jina-V3 & \textbf{94.90} & \textbf{94.90} & 64.89 & 64.81 & 2.40 & 2.21 & \textbf{29.95} & \textbf{33.88} & 62.54 & 62.54 & 92.61 & 92.61 & 63.08 & 63.13 \\
& Nomic & 76.88 & 76.88 & 55.83 & 56.43 & \textbf{13.71} & \textbf{13.90} & 7.72 & 4.66 & 36.83 & 36.83 & 72.83 & 72.83 & 51.87 & 52.26 \\
& Qwen3-0.6B & 93.99 & 93.99 & \textbf{70.47} & \textbf{70.31} & 5.73 & 6.43 & 13.17 & 28.79 & \textbf{66.71} & \textbf{66.71} & \textbf{94.18} & \textbf{94.18} & \textbf{67.52} & \textbf{67.89} \\
\bottomrule
\end{tabular}
\end{sidewaystable}
\clearpage
\begin{table}[htbp!]
\centering
\footnotesize
\caption{Summary of Average Performance Across All Datasets (in \%). Best result per column within each k-group is in \textbf{{bold}}.}
\label{tab:summary}
\begin{tabular}{llcccccccc}
\toprule
\multirow{2}{*}{k} & \multirow{2}{*}{Model} & \multicolumn{2}{c}{RECALL} & \multicolumn{2}{c}{PRECISION} & \multicolumn{2}{c}{MRR} & \multicolumn{2}{c}{NDCG} \\
\cmidrule(lr){3-4} \cmidrule(lr){5-6} \cmidrule(lr){7-8} \cmidrule(lr){9-10}
& & L2 & IP & L2 & IP & L2 & IP & L2 & IP \\
\midrule
\multirow{5}{*}{1} & Ada-002 & 52.01 & 51.98 & 52.01 & 51.98 & 52.01 & 51.98 & 52.01 & 51.98 \\
& Jina-V2 & 41.48 & 4.66 & 41.48 & 4.66 & 41.48 & 4.66 & 41.48 & 4.66 \\
& Jina-V3 & 54.69 & 54.67 & 54.69 & 54.67 & 54.69 & 54.67 & 54.69 & 54.67 \\
& Nomic & 42.67 & 43.13 & 42.67 & 43.13 & 42.67 & 43.13 & 42.67 & 43.13 \\
& Qwen3-0.6B & \textbf{59.45} & \textbf{59.45} & \textbf{59.45} & \textbf{59.45} & \textbf{59.45} & \textbf{59.45} & \textbf{59.45} & \textbf{59.45} \\
\midrule
\multirow{5}{*}{5} & Ada-002 & 62.71 & 62.66 & 12.54 & 12.53 & 56.17 & 56.08 & 57.81 & 57.73 \\
& Jina-V2 & 51.85 & 12.12 & 10.37 & 2.42 & 45.47 & 7.31 & 47.07 & 8.50 \\
& Jina-V3 & 66.09 & 66.22 & 13.22 & 13.24 & 59.09 & 59.11 & 60.84 & 60.89 \\
& Nomic & 54.69 & 55.11 & 10.94 & 11.02 & 47.30 & 47.75 & 49.15 & 49.59 \\
& Qwen3-0.6B & \textbf{70.55} & \textbf{71.23} & \textbf{14.11} & \textbf{14.25} & \textbf{63.77} & \textbf{64.02} & \textbf{65.47} & \textbf{65.82} \\
\midrule
\multirow{5}{*}{10} & Ada-002 & 65.91 & 65.94 & 6.59 & 6.59 & 56.61 & 56.53 & 58.86 & 58.80 \\
& Jina-V2 & 55.58 & 15.43 & 5.56 & 1.54 & 45.97 & 7.76 & 48.28 & 9.58 \\
& Jina-V3 & 69.94 & 70.13 & 6.99 & 7.01 & 59.61 & 59.64 & 62.10 & 62.16 \\
& Nomic & 58.92 & 59.24 & 5.89 & 5.92 & 47.87 & 48.30 & 50.53 & 50.93 \\
& Qwen3-0.6B & \textbf{74.00} & \textbf{74.58} & \textbf{7.40} & \textbf{7.46} & \textbf{64.24} & \textbf{64.47} & \textbf{66.59} & \textbf{66.91} \\
\midrule
\multirow{5}{*}{25} & Ada-002 & 69.74 & 69.93 & 2.79 & 2.80 & 56.86 & 56.79 & 59.80 & 59.78 \\
& Jina-V2 & 59.63 & 21.37 & 2.39 & 0.85 & 46.23 & 8.12 & 49.27 & 11.02 \\
& Jina-V3 & 73.92 & 74.08 & 2.96 & 2.96 & 59.87 & 59.89 & 63.08 & 63.13 \\
& Nomic & 64.41 & 64.70 & 2.58 & 2.59 & 48.22 & 48.64 & 51.87 & 52.26 \\
& Qwen3-0.6B & \textbf{77.79} & \textbf{78.53} & \textbf{3.11} & \textbf{3.14} & \textbf{64.48} & \textbf{64.73} & \textbf{67.52} & \textbf{67.89} \\
\bottomrule
\end{tabular}
\end{table}

The results show that the \textbf{Qwen3-Embedding-0.6B} model is the strongest performer overall, achieving the highest average recall, precision, MRR, and nDCG across all k-values. This suggests that it is a highly effective and robust general-purpose retriever. The \textbf{jina-embeddings-v3} model, with its late chunking strategy, also performs well, especially on the `passkey` dataset. The \textbf{nomic-embed-text} model excels at the `needle` task, indicating its strength in finding specific facts in long documents. The \textbf{ada-002} and \textbf{jina-embeddings-v2-base-en} models show competitive performance on some datasets but do not consistently match the top performers.


\section{Experiment 2: Reranking Strategies}
\label{sec:exp_reranking}
This experiment evaluated the benefit of adding a reranking step after the initial retrieval (using Sentence-BERT).
\subsection{Reranking Methods Compared}
\begin{itemize}
    \item \textbf{No Reranking (Baseline)}.
    \item \textbf{BM25 Reranking:} Using BM25 scores to rerank the top 20 candidates from the initial dense retrieval.
    \item \textbf{Cross-Encoder Reranking:} Using a powerful cross-encoder model to rerank the top 20 candidates.
\end{itemize}
\subsection{Results and Discussion}
The results in Table \ref{tab:reranking_results} demonstrate the significant impact of reranking. The BM25 reranker provided a solid improvement by adding a keyword-based signal. However, the cross-encoder, despite its higher computational cost, yielded the best performance by a clear margin, showcasing the power of deep, token-level comparison for fine-grained relevance assessment.

\begin{table}[!htbp]
\centering
\caption{Reranking Strategy Performance}
\label{tab:reranking_results}
\begin{tabular}{|l|c|c|}
\hline
\textbf{Method} & \textbf{Precision@5} & \textbf{MRR} \\
\hline
No Reranking & TBD & TBD \\
BM25 Reranking & TBD & TBD \\
Cross-Encoder & \textbf{TBD} & \textbf{TBD} \\
\hline
\end{tabular}
\end{table}

\section{Experiment 3: Generative Model and Prompt Evaluation}
\label{sec:exp_generator_prompt}
This experiment focuses on the generation part of the RAG pipeline, evaluating a wide range of Large Language Models to serve as the generator. The primary goal was to find the best-performing model and the optimal prompt and temperature configuration for our specific use case.

\subsection{Motivation and Baseline}
The evaluation started with a baseline configuration of \textbf{GPT-4o} using prompt \texttt{P1} and a temperature of \texttt{0.2}. As new and more powerful models were released during the course of this research, they were systematically evaluated against this baseline. This continuous evaluation process allowed us to stay at the cutting edge and select the most suitable model for integration into a production RAG system, ensuring that any replacement would offer superior performance.

\subsection{Evaluation Methodology}
We used the \textbf{DeepEval} framework \autocite{deepeval2023}, which employs an LLM-as-a-Judge approach to score the generator's output. For subjective, use-case-specific evaluations, we utilized DeepEval's \textbf{G-Eval} functionality, which is inspired by the G-Eval framework \autocite{liu2023geval} and uses a chain-of-thought process to evaluate outputs against custom criteria. We defined two such metrics:
\begin{itemize}
    \item \textbf{Correctness (C):} This metric determines whether the actual output is factually correct based on the expected output. It heavily penalizes the omission of key details and the inclusion of information that contradicts the expected output.
    \item \textbf{Specific Information Accuracy (SIA):} This metric evaluates whether the model's response appropriately uses information from the context without introducing specific details (names, places, numbers) that are not explicitly provided. It is particularly important for testing the model's ability to recognize when a question cannot be answered from the given context. For example, a high SIA score is awarded if the model correctly states it cannot answer a question like "Who is Cinderella?" when given an anonymized version of the story, as this demonstrates it is not relying on its own parametric knowledge.
\end{itemize}

In addition to our custom G-Eval metrics, we used several of DeepEval's standard metrics to assess other critical qualities of the generated output:
\begin{itemize}
    \item \textbf{Answer Relevancy (AR):} This metric measures how relevant the generated output is to the user's input query. It ensures that the model's response is on-topic and directly addresses the question asked \autocite{deepeval2023}.
    \item \textbf{Faithfulness (F):} This metric evaluates whether the generated output factually aligns with the information present in the retrieved context. A high score indicates that the model did not invent facts and stayed true to the source material \autocite{deepeval2023}.
    \item \textbf{Hallucination (H):} This metric determines if the model generates factually incorrect information by comparing the output to the provided context. It is a crucial measure for ensuring the trustworthiness of the RAG system \autocite{deepeval2023}.
\end{itemize}


\subsection{Results and Discussion}
The comprehensive results are presented in Table \ref{tab:deepeval_results}. The baseline configuration (GPT-4o, P1, Temp 0.2) achieved a total score of 370.52. The results show that several newer models and prompt configurations were able to significantly outperform this baseline.

For instance, the \textbf{O4 Mini} model using prompt \texttt{P2} and a temperature of \texttt{0.2} achieved the highest total score of \textbf{403.86}. This demonstrates the value of continuous evaluation, as it allowed us to identify a model that not only performs better than the original baseline but also to determine the ideal prompt (\texttt{P2}) and temperature (\texttt{0.2}) for it. These findings are critical for deploying a RAG system that is not only accurate and faithful but also robust in handling queries that cannot be answered from the available context.

\begin{landscape}
    \footnotesize
    \setlength{\tabcolsep}{3pt}       % default is ~6pt
    \renewcommand{\arraystretch}{0.9}  % tighten row height
    \setlength{\LTleft}{0pt}          % left align longtable
    \setlength{\LTright}{0pt}         % right align longtable
    \begin{longtable}{|l|c|c|ccccc|c|ccccc|ccccc|}
        \caption{DeepEval Generative Model and Prompt Evaluation Results}
        \label{tab:deepeval_results} \\

        \toprule
        \textbf{Model} & \textbf{Prompt} & \textbf{Temp} & \multicolumn{5}{c|}{\textbf{Avg. Scores}} & \textbf{Total} & \multicolumn{5}{c|}{\textbf{GPT-4o}} & \multicolumn{5}{c|}{\textbf{Claude 3.5}} \\
        & & & \textbf{AR} & \textbf{C} & \textbf{F} & \textbf{H} & \textbf{SIA} & & \textbf{AR} & \textbf{C} & \textbf{F} & \textbf{H} & \textbf{SIA} & \textbf{AR} & \textbf{C} & \textbf{F} & \textbf{H} & \textbf{SIA} \\
        \midrule
        \endfirsthead

        \multicolumn{19}{c}{\tablename\ \thetable{} -- continued from previous page} \\
        \midrule
        \textbf{Model} & \textbf{Prompt} & \textbf{Temp} & \textbf{AR} & \textbf{C} & \textbf{F} & \textbf{H} & \textbf{SIA} & \textbf{Total} & \textbf{AR} & \textbf{C} & \textbf{F} & \textbf{H} & \textbf{SIA} & \textbf{AR} & \textbf{C} & \textbf{F} & \textbf{H} & \textbf{SIA} \\
        \midrule
        \endhead

        \midrule
        \multicolumn{19}{l}{\footnotesize{\textbf{AR}: Answer Relevancy, \textbf{C}: Correctness, \textbf{F}: Faithfulness, \textbf{H}: Hallucination, \textbf{SIA}: Specific Information Accuracy}} \\
        \multicolumn{19}{r}{Continued on next page} \\ 
        \endfoot

        \bottomrule
        \multicolumn{19}{l}{\footnotesize{\textbf{AR}: Answer Relevancy, \textbf{C}: Correctness, \textbf{F}: Faithfulness, \textbf{H}: Hallucination, \textbf{SIA}: Specific Information Accuracy}}
        \endlastfoot
        Claude 3 Haiku & P1 & 0.0 & 57.70 & 66.66 & 87.18 & 69.23 & 74.36 & 355.13 & 61.54 & 69.23 & 84.62 & 64.10 & 79.49 & 53.85 & 64.10 & 89.74 & 74.36 & 69.23 \\
        Claude 3 Haiku & P1 & 0.2 & 62.82 & 61.54 & 87.18 & 65.38 & 67.94 & 344.86 & 61.54 & 53.85 & 84.62 & 58.97 & 64.10 & 64.10 & 69.23 & 89.74 & 71.79 & 71.79 \\
        Claude 3 Haiku & P2 & 0.0 & 66.66 & 62.82 & 89.74 & 67.94 & 74.36 & 361.52 & 64.10 & 56.41 & 89.74 & 64.10 & 82.05 & 69.23 & 69.23 & 89.74 & 71.79 & 66.67 \\
        Claude 3 Haiku & P2 & 0.2 & 62.82 & 61.53 & 79.49 & 65.38 & 73.07 & 342.29 & 61.54 & 58.97 & 79.49 & 58.97 & 82.05 & 64.10 & 64.10 & 79.49 & 71.79 & 64.10 \\
        Claude 3.5 Sonnet & P1 & 0.0 & 58.97 & 76.93 & 92.31 & 74.35 & 71.79 & 374.35 & 61.54 & 69.23 & 97.44 & 58.97 & 71.79 & 56.41 & 84.62 & 87.18 & 89.74 & 71.79 \\
        Claude 3.5 Sonnet v2 & P1 & 0.0 & 48.72 & 75.65 & 96.16 & 66.66 & 67.95 & 355.14 & 48.72 & 66.67 & 92.31 & 51.28 & 69.23 & 48.72 & 84.62 & 100.00 & 82.05 & 66.67 \\
        Claude 3.5 Sonnet v2 & P1 & 0.2 & 52.56 & 69.23 & 96.16 & 71.80 & 69.23 & 358.98 & 53.85 & 53.85 & 94.87 & 56.41 & 76.92 & 51.28 & 84.62 & 97.44 & 87.18 & 61.54 \\
        Claude 3.5 Sonnet v2 & P2 & 0.0 & 62.82 & 70.52 & 91.03 & 78.20 & 84.62 & 387.19 & 64.10 & 61.54 & 89.74 & 61.54 & 84.62 & 61.54 & 79.49 & 92.31 & 94.87 & 84.62 \\
        Claude 3.5 Sonnet v2 & P2 & 0.2 & 64.10 & 75.64 & 94.87 & 76.92 & 88.46 & 399.99 & 71.79 & 74.36 & 94.87 & 61.54 & 89.74 & 56.41 & 76.92 & 94.87 & 92.31 & 87.18 \\
        Claude 3.7 Sonnet & P1 & 0.0 & 48.72 & 82.05 & 88.47 & 71.80 & 78.20 & 369.24 & 48.72 & 69.23 & 92.31 & 56.41 & 74.36 & 48.72 & 94.87 & 84.62 & 87.18 & 82.05 \\
        Claude 3.7 Sonnet & P1 & 0.2 & 44.87 & 88.46 & 88.46 & 73.08 & 88.46 & 383.33 & 46.15 & 87.18 & 94.87 & 58.97 & 94.87 & 43.59 & 89.74 & 82.05 & 87.18 & 82.05 \\
        Claude 3.7 Sonnet & P2 & 0.0 & 48.72 & 78.21 & 93.59 & 78.20 & 85.90 & 384.62 & 53.85 & 71.79 & 94.87 & 61.54 & 79.49 & 43.59 & 84.62 & 92.31 & 94.87 & 92.31 \\
        Claude 3.7 Sonnet & P2 & 0.2 & 43.59 & 84.62 & 89.74 & 79.49 & 92.31 & 389.75 & 46.15 & 82.05 & 89.74 & 66.67 & 92.31 & 41.03 & 87.18 & 89.74 & 92.31 & 92.31 \\
        Claude 4.0 Sonnet & P1 & 0.2 & 38.46 & 83.34 & 89.75 & 76.92 & 83.33 & 371.80 & 38.46 & 79.49 & 87.18 & 61.54 & 89.74 & 38.46 & 87.18 & 92.31 & 92.31 & 76.92 \\
        Claude 4.0 Sonnet & P2 & 0.2 & 42.31 & 84.62 & 91.03 & 79.49 & 91.03 & 388.48 & 41.03 & 84.62 & 97.44 & 66.67 & 94.87 & 43.59 & 84.62 & 84.62 & 92.31 & 87.18 \\
        GPT-4 Omni & P1 & 0.0 & 56.41 & 65.38 & 93.59 & 70.52 & 67.94 & 353.84 & 53.85 & 58.97 & 92.31 & 56.41 & 71.79 & 58.97 & 71.79 & 94.87 & 84.62 & 64.10 \\
        GPT-4 Omni & P1 & 0.2 & 60.26 & 78.20 & 92.31 & 67.95 & 71.80 & 370.52 & 53.85 & 74.36 & 94.87 & 51.28 & 69.23 & 66.67 & 82.05 & 89.74 & 84.62 & 74.36 \\
        GPT-4 Omni & P2 & 0.0 & 56.41 & 74.36 & 92.31 & 61.54 & 78.20 & 362.82 & 51.28 & 64.10 & 92.31 & 48.72 & 76.92 & 61.54 & 84.62 & 92.31 & 74.36 & 79.49 \\
        GPT-4 Omni & P2 & 0.2 & 58.97 & 73.08 & 92.31 & 67.95 & 79.48 & 371.79 & 53.85 & 61.54 & 92.31 & 56.41 & 82.05 & 64.10 & 84.62 & 92.31 & 79.49 & 76.92 \\
        GPT-4 Omni Mini & P1 & 0.0 & 61.54 & 73.08 & 91.03 & 70.52 & 66.66 & 362.83 & 51.28 & 71.79 & 89.74 & 56.41 & 69.23 & 71.79 & 74.36 & 92.31 & 84.62 & 64.10 \\
        GPT-4 Omni Mini & P1 & 0.2 & 61.54 & 71.80 & 94.88 & 69.23 & 70.52 & 367.97 & 56.41 & 69.23 & 92.31 & 53.85 & 74.36 & 66.67 & 74.36 & 97.44 & 84.62 & 66.67 \\
        GPT-4 Omni Mini & P2 & 0.0 & 65.38 & 74.36 & 92.31 & 57.69 & 75.64 & 365.38 & 66.67 & 69.23 & 92.31 & 43.59 & 79.49 & 64.10 & 79.49 & 92.31 & 71.79 & 71.79 \\
        GPT-4 Omni Mini & P2 & 0.2 & 64.10 & 76.92 & 91.03 & 58.97 & 80.77 & 371.79 & 64.10 & 71.79 & 87.18 & 46.15 & 82.05 & 64.10 & 82.05 & 94.87 & 71.79 & 79.49 \\
        GPT-4.1 & P1 & 0.2 & 65.38 & 82.05 & 93.59 & 73.07 & 85.89 & 399.98 & 64.10 & 82.05 & 97.44 & 56.41 & 89.74 & 66.67 & 82.05 & 89.74 & 89.74 & 82.05 \\
        GPT-4.1 & P2 & 0.2 & 62.82 & 80.77 & 93.59 & 70.52 & 89.74 & 397.44 & 61.54 & 76.92 & 94.87 & 61.54 & 89.74 & 64.10 & 84.62 & 92.31 & 79.49 & 89.74 \\
        GPT-4.1 Nano & P1 & 0.2 & 56.41 & 69.23 & 93.59 & 75.64 & 73.08 & 367.95 & 58.97 & 61.54 & 92.31 & 58.97 & 71.79 & 53.85 & 76.92 & 94.87 & 92.31 & 74.36 \\
        GPT-4.1 Nano & P2 & 0.2 & 65.38 & 57.69 & 87.18 & 69.23 & 78.20 & 357.68 & 64.10 & 51.28 & 89.74 & 58.97 & 79.49 & 66.67 & 64.10 & 84.62 & 79.49 & 76.92 \\
        Gemini 1.5 Flash & P1 & 0.0 & 73.08 & 61.53 & 93.59 & 62.82 & 65.39 & 356.41 & 69.23 & 58.97 & 92.31 & 51.28 & 61.54 & 76.92 & 64.10 & 94.87 & 74.36 & 69.23 \\
        Gemini 1.5 Flash & P1 & 0.2 & 73.08 & 57.69 & 96.16 & 61.54 & 67.95 & 356.42 & 71.79 & 51.28 & 94.87 & 48.72 & 69.23 & 74.36 & 64.10 & 97.44 & 74.36 & 66.67 \\
        Gemini 1.5 Flash & P2 & 0.0 & 78.20 & 52.56 & 96.16 & 69.23 & 67.95 & 364.10 & 76.92 & 51.28 & 97.44 & 53.85 & 66.67 & 79.49 & 53.85 & 94.87 & 84.62 & 69.23 \\
        Gemini 1.5 Flash & P2 & 0.2 & 78.20 & 67.95 & 92.31 & 69.23 & 82.06 & 389.75 & 82.05 & 66.67 & 92.31 & 53.85 & 79.49 & 74.36 & 69.23 & 92.31 & 84.62 & 84.62 \\
        Gemini 1.5 Pro & P1 & 0.0 & 74.36 & 60.26 & 93.59 & 65.39 & 56.41 & 350.01 & 74.36 & 53.85 & 100.00 & 43.59 & 53.85 & 74.36 & 66.67 & 87.18 & 87.18 & 58.97 \\
        Gemini 1.5 Pro & P1 & 0.2 & 75.64 & 52.56 & 91.03 & 64.11 & 55.13 & 338.47 & 74.36 & 48.72 & 94.87 & 43.59 & 56.41 & 76.92 & 56.41 & 87.18 & 84.62 & 53.85 \\
        Gemini 1.5 Pro & P2 & 0.0 & 67.94 & 56.41 & 97.44 & 65.38 & 65.38 & 352.55 & 64.10 & 51.28 & 97.44 & 51.28 & 71.79 & 71.79 & 61.54 & 97.44 & 79.49 & 58.97 \\
        Gemini 1.5 Pro & P2 & 0.2 & 74.36 & 58.97 & 93.59 & 60.25 & 74.36 & 361.53 & 71.79 & 58.97 & 92.31 & 46.15 & 74.36 & 76.92 & 58.97 & 94.87 & 74.36 & 74.36 \\
        Gemini 2.0 Flash & P1 & 0.0 & 48.72 & 48.72 & 97.44 & 65.39 & 53.84 & 314.11 & 51.28 & 46.15 & 97.44 & 43.59 & 48.72 & 46.15 & 51.28 & 97.44 & 87.18 & 58.97 \\
        Gemini 2.0 Flash & P1 & 0.2 & 47.44 & 46.16 & 100.00 & 58.98 & 56.41 & 308.99 & 46.15 & 41.03 & 100.00 & 33.33 & 48.72 & 48.72 & 51.28 & 100.00 & 84.62 & 64.10 \\
        Gemini 2.0 Flash & P2 & 0.0 & 66.66 & 61.53 & 93.59 & 67.95 & 74.36 & 364.09 & 64.10 & 58.97 & 92.31 & 56.41 & 71.79 & 69.23 & 64.10 & 94.87 & 79.49 & 76.92 \\
        Gemini 2.0 Flash & P2 & 0.2 & 73.08 & 56.41 & 92.31 & 65.38 & 75.64 & 362.82 & 71.79 & 51.28 & 92.31 & 51.28 & 74.36 & 74.36 & 61.54 & 92.31 & 79.49 & 76.92 \\
        Gemini 2.5 Flash & P2 & 0.2 & 55.12 & 74.36 & 91.03 & 69.23 & 85.90 & 375.64 & 51.28 & 66.67 & 92.31 & 56.41 & 87.18 & 58.97 & 82.05 & 89.74 & 82.05 & 84.62 \\
        Gemini 2.5 Flash & P2 & 0.2 & 57.70 & 75.64 & 91.03 & 74.35 & 87.18 & 385.90 & 53.85 & 74.36 & 94.87 & 58.97 & 89.74 & 61.54 & 76.92 & 87.18 & 89.74 & 84.62 \\
        Gemini 2.5 Flash & P2 & 0.2 & 62.82 & 78.20 & 91.03 & 70.52 & 87.18 & 389.75 & 58.97 & 76.92 & 92.31 & 56.41 & 89.74 & 66.67 & 79.49 & 89.74 & 84.62 & 84.62 \\
        Gemini 2.5 Flash & P2 & 0.2 & 66.66 & 74.36 & 89.75 & 69.23 & 88.46 & 388.46 & 58.97 & 66.67 & 92.31 & 53.85 & 87.18 & 74.36 & 82.05 & 87.18 & 84.62 & 89.74 \\
        Gemini 2.5 Pro & P2 & 0.2 & 62.82 & 82.05 & 93.59 & 73.08 & 91.03 & 402.57 & 61.54 & 74.36 & 94.87 & 61.54 & 94.87 & 64.10 & 89.74 & 92.31 & 84.62 & 87.18 \\
        Gemini 2.5 Pro & P2 & 0.2 & 62.82 & 82.05 & 89.74 & 71.80 & 88.46 & 394.87 & 66.67 & 71.79 & 89.74 & 58.97 & 89.74 & 58.97 & 92.31 & 89.74 & 84.62 & 87.18 \\
        Gemini 2.5 Pro & P2 & 0.2 & 57.69 & 76.93 & 92.31 & 75.64 & 87.18 & 389.75 & 58.97 & 69.23 & 97.44 & 61.54 & 87.18 & 56.41 & 84.62 & 87.18 & 89.74 & 87.18 \\
        Gemini 2.5 Pro & P2 & 0.2 & 62.82 & 78.21 & 88.47 & 76.92 & 87.18 & 393.60 & 58.97 & 69.23 & 92.31 & 64.10 & 89.74 & 66.67 & 87.18 & 84.62 & 89.74 & 84.62 \\
        O1 & P1 & 0.0 & 66.67 & 82.05 & 93.59 & 76.92 & 84.62 & 403.85 & 71.79 & 76.92 & 94.87 & 56.41 & 87.18 & 61.54 & 87.18 & 92.31 & 97.44 & 82.05 \\
        O1 & P2 & 0.0 & 70.52 & 71.79 & 97.44 & 62.82 & 93.59 & 396.16 & 66.67 & 64.10 & 97.44 & 46.15 & 92.31 & 74.36 & 79.49 & 97.44 & 79.49 & 94.87 \\
        O3 & P1 & 0.2 & 75.64 & 66.67 & 88.47 & 69.23 & 78.20 & 378.21 & 76.92 & 74.36 & 61.54 & 71.79 & 84.62 & 92.31 & 53.85 & 84.62 & 76.92 & 79.49 \\
        O3 & P2 & 0.2 & 70.51 & 80.77 & 91.03 & 73.07 & 85.89 & 401.27 & 71.79 & 71.79 & 89.74 & 56.41 & 82.05 & 69.23 & 89.74 & 92.31 & 89.74 & 89.74 \\
        O3 Mini & P1 & 0.0 & 70.52 & 67.95 & 96.16 & 74.36 & 62.82 & 371.81 & 74.36 & 56.41 & 94.87 & 61.54 & 66.67 & 66.67 & 79.49 & 97.44 & 87.18 & 58.97 \\
        O3 Mini & P1 & 0.2 & 60.25 & 65.38 & 93.59 & 67.95 & 65.39 & 352.56 & 64.10 & 48.72 & 94.87 & 51.28 & 61.54 & 56.41 & 82.05 & 92.31 & 84.62 & 69.23 \\
        O3 Mini & P2 & 0.0 & 64.11 & 74.36 & 92.31 & 74.36 & 75.64 & 380.78 & 61.54 & 69.23 & 89.74 & 66.67 & 74.36 & 66.67 & 79.49 & 94.87 & 82.05 & 76.92 \\
        O3 Mini & P2 & 0.2 & 65.38 & 71.80 & 96.16 & 69.23 & 79.49 & 382.06 & 66.67 & 61.54 & 94.87 & 56.41 & 79.49 & 64.10 & 82.05 & 97.44 & 82.05 & 79.49 \\
        O4 Mini & P1 & 0.2 & 67.95 & 75.64 & 93.59 & 73.07 & 83.34 & 393.59 & 69.23 & 66.67 & 76.92 & 74.36 & 94.87 & 92.31 & 56.41 & 89.74 & 84.62 & 82.05 \\
        O4 Mini & P2 & 0.2 & 74.36 & 83.34 & 93.59 & 66.67 & 85.90 & 403.86 & 71.79 & 79.49 & 92.31 & 48.72 & 87.18 & 76.92 & 87.18 & 94.87 & 84.62 & 84.62 \\
    \end{longtable}
\end{landscape}

\subsection{Comparative Analysis Against GPT-4o}
\label{sec:comparative_analysis_gpt4o}

To better understand the relative performance of each model, we calculated the delta between each model's total score and the baseline score of GPT-4o (370.52). The results are summarized in Table \ref{tab:deepeval_delta_results}.

\begin{table}[!htbp]
\centering
\small
\caption{Total Score Delta vs. GPT-4o (Highest Scores)}
\label{tab:deepeval_delta_results}
\begin{tabular}{|l|c|c|c|}
\hline
\textbf{Model} & \textbf{Prompt} & \textbf{Temp} & \textbf{Delta from GPT-4o} \\
\hline
O4 Mini & P2 & 0.2 & +33.34 \\
O1 & P1 & 0.0 & +33.33 \\
Gemini 2.5 Pro & P2 & 0.2 & +32.05 \\
O3 & P2 & 0.2 & +30.75 \\
Claude 3.5 Sonnet v2 & P2 & 0.2 & +29.47 \\
GPT-4.1 & P1 & 0.2 & +29.46 \\
Claude 3.7 Sonnet & P2 & 0.2 & +19.23 \\
Gemini 1.5 Flash & P2 & 0.2 & +19.23 \\
Gemini 2.5 Flash & P2 & 0.2 & +19.23 \\
Claude 4.0 Sonnet & P2 & 0.2 & +17.96 \\
O3 Mini & P2 & 0.2 & +11.54 \\
Claude 3.5 Sonnet & P1 & 0.0 & +3.83 \\
GPT-4 Omni Mini & P2 & 0.2 & +1.27 \\
GPT-4 Omni & P2 & 0.2 & +1.27 \\
\hline
\multicolumn{4}{|c|}{\textbf{GPT-4o Baseline Total Score: 370.52}} \\
\hline
GPT-4.1 Nano & P1 & 0.2 & -2.57 \\
Gemini 2.0 Flash & P2 & 0.0 & -6.43 \\
Claude 3 Haiku & P2 & 0.0 & -9.00 \\
Gemini 1.5 Pro & P2 & 0.2 & -9.00 \\
\hline
\end{tabular}
\end{table}

\subsubsection{Inference from the Results}
The delta analysis reveals several key insights. Firstly, the top-performing models, such as \textbf{O4 Mini}, \textbf{O1}, and \textbf{Gemini 2.5 Pro}, show a significant improvement over the GPT-4o baseline, with score increases exceeding 30 points. This underscores the rapid advancements in generative models, where newer architectures can yield substantial performance gains.

Secondly, the choice of prompt and temperature settings is evidently crucial. For many models, there is a considerable performance variance between different prompt versions (P1 vs. P2) and temperature settings. For instance, the O1 model with prompt P1 scores much higher than with P2 (403.85 vs 396.16: +7.69 points). This highlights the necessity of meticulous prompt engineering and parameter tuning to unlock a model's full potential.


\section{Overall Performance Summary and Analysis}
\label{sec:overall_analysis}
By combining the best-performing components from each experiment, we constructed an optimized RAG pipeline: Semantic Chunking (Experiment \ref{sec:exp_chunking}) + Sentence-BERT (Experiment \ref{sec:exp_embedding_models}) + Cross-Encoder Reranking (Experiment \ref{sec:exp_reranking}) + the best performing model from the DeepEval benchmark (Experiment \ref{sec:exp_generator_prompt}). As shown in Table \ref{tab:overall_results}, this optimized configuration dramatically outperformed the initial baseline across all key metrics. This demonstrates that a systematic, component-wise optimization of the RAG pipeline can lead to substantial gains in both retrieval accuracy and generation quality, resulting in a more robust and trustworthy system.

\begin{table}[!htbp]
\centering
\caption{Overall Performance: Baseline vs. Optimized}
\label{tab:overall_results}
\begin{tabular}{|l|c|c|c|}
\hline
\textbf{Configuration} & \textbf{Precision@5} & \textbf{Faithfulness} & \textbf{Hallucination Rate} \\
\hline
Baseline & TBD & TBD & TBD \\
Optimized & \textbf{TBD} & \textbf{TBD} & \textbf{TBD} \\
\hline
\end{tabular}
\end{table}