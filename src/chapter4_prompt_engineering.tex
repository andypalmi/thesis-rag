\chapter{Prompt Engineering for RAG}
\label{chap:prompt_engineering}

In a Retrieval-Augmented Generation system, the prompt serves as the crucial interface between the retrieved information and the generative capabilities of the Large Language Model. Effective prompt engineering is indispensable for guiding the LLM in synthesizing the provided context and generating an accurate, relevant, and coherently structured response. This chapter explores the fundamental principles of prompt design for RAG, different prompting styles, and the methodologies for evaluating their effectiveness.

\section{Concepts of Prompt Engineering}
A RAG prompt exhibits greater complexity compared to a standard query directed at an LLM. It must effectively integrate two distinct informational components: the user's original query and the retrieved context. The structure of the prompt governs the LLM's interpretation and utilization of the provided information.

A typical RAG prompt includes:
\begin{itemize}
    \item \textbf{Instructions:} Explicit directives to the LLM regarding its operational behavior. This may encompass instructions to exclusively utilize the provided context, to assume a particular persona, or to format the output in a specified manner.
    \item \textbf{Context:} The set of retrieved document chunks that are relevant to the query.
    \item \textbf{Query:} The user's original question or request.
\end{itemize}

The formulation of the prompt necessitates a meticulous arrangement of these elements. For instance, a typical template might be structured as follows:

\begin{tcolorbox}[promptbox,title=Example: A Typical RAG prompt]
You are a helpful assistant. Use the following context to answer the question at the end.\\
If you don't know the answer, just say that you don't know, don't try to make up an answer.\\

Context: \{retrieved chunks\}\\

Question: \{user query\}
\end{tcolorbox}

\section{Prompt Styles}
Diverse prompting styles can be employed contingent upon the specific task and the LLM in use. The selection of style can substantially influence the quality of the generated response. Various prompting techniques exist, each with its own strengths and applications, allowing for fine-grained control over the LLM's behavior \autocite{liu2023geval}.

\subsubsection{Zero-Shot Prompting}
This represents the most prevalent style in RAG, wherein the LLM receives instructions and context without prior examples demonstrating task completion \autocite{brown2020language}. The efficacy of zero-shot prompts is highly dependent on the clarity of the instructions and the LLM's intrinsic capacity for reasoning and adherence to directives.


\begin{tcolorbox}[promptbox,title=Example: Zero-Shot Prompting]
You are a helpful assistant. Use the following context to answer the question at the end.\\
If you don't know the answer, just say that you don't know, don't try to make up an answer.\\

Context: \{retrieved chunks\}\\

Question: \{user query\}
\end{tcolorbox}
    
\subsubsection{Few-Shot Prompting}
In this approach, the prompt incorporates a limited number of examples (shots) illustrating the desired input-output format \autocite{brown2020language}. This can prove particularly advantageous for intricate tasks or when a highly specific output structure is mandated. For RAG, this might entail demonstrating to the model how to synthesize context into an answer for a few sample questions prior to presenting the actual query.

\begin{tcolorbox}[promptbox,title=Example: Few Shot Prompting]
You are a helpful assistant. Use the following context to answer the question at the end.\\
If you don't know the answer, just say that you don't know, don't try to make up an answer.\\

Here are a few examples:\\

Context: "The capital of France is Paris."\\
Question: "What is the capital of France?"\\
Answer: "The capital of France is Paris."\\\\
Context: "Mount Everest is 8848 meters tall."\\
Question: "What is the highest mountain in the world?"\\
Answer: "I'm sorry, I cannot answer with the provided information."\\

Context: \{retrieved chunks\}

Question: \{user query\}
\end{tcolorbox}
    
\subsubsection{Instruction-Based Prompts}
These prompts emphasize the provision of highly detailed and explicit instructions \autocite{zhang2025instructiontuninglargelanguage}. This may encompass negative constraints (e.g., "Do not mention information not present in the context") and positive constraints (e.g., "Summarize the answer in three bullet points").

\begin{tcolorbox}[promptbox,title=Example: Instruction-Based Prompt]
You are a concise summarizer.\\
Based on the following context, provide a summary of the key points in exactly three bullet points.\\
Do not include any information not explicitly stated in the context.\\

Context: \{retrieved chunks\}\\

Question: \{user query\}
\end{tcolorbox}
    
\subsubsection{Role-Playing Prompts}
Assigning a specific role to the LLM (e.g., "You are a financial analyst reviewing these documents") can assist in framing the context and directing the tone and focus of the response \autocite{tseng-etal-2024-two}.

\begin{tcolorbox}[promptbox,title=Example: Role-Playing Prompt]
You are a seasoned medical professional.\\
Analyze the provided patient records and summarize the key health concerns and recommended treatments.\\
Ensure your response is empathetic and easy for a non-medical person to understand.\\

Context: \{retrieved chunks\}\\

Question: \{user query\}
\end{tcolorbox}

\section{Prompt Evaluation}
Evaluating the efficacy of different prompts constitutes a critical step in optimizing a RAG pipeline. Given that the quality of a generated response can be subjective, a combination of qualitative and quantitative methods is frequently employed.

\subsection{Golden Datasets}
A \textbf{golden dataset} comprises a curated collection of queries coupled with optimal, human-verified answers. For our specific evaluation, this dataset was meticulously created internally within the company and consists of 50 samples. A crucial aspect of this dataset is its inclusion of examples designed to test the model's ability to refrain from answering when appropriate. This includes scenarios where the provided context is wrong, empty, or, more subtly, where the key information being asked is missing, even if other related information is present.

For a business case where the factuality of the model's answer is of the utmost importance, the dataset contains anonymized versions of famous fables: Cinderella and Snow White. When presented with questions like "Who is Cinderella?" or "Who is Snow White?", the model is expected to refrain from answering, as the protagonist's name is intentionally omitted from the context, even though the model might be trained on the general story of these fables.

Other examples within the dataset are designed to evaluate the model's capability of finding a key piece of information in a large document. Furthermore, many cases focus on the model's ability to answer a question using only the information present in the context. This capability of the LLM to answer solely based on provided context was an important business requirement that was thoroughly studied and evaluated. The answers generated by the LLM were evaluated against these golden answers using DeepEval as the framework, which leverages the LLM-as-a-Judge methodology for these evaluations. By comparing the LLM's output for a given query against the golden answer, one can evaluate the quality of the prompt, furnishing a robust benchmark for evaluation.

\subsection{LLM-as-a-Judge}
A more scalable approach to evaluation involves employing another, often more powerful, LLM as an evaluator, a methodology demonstrated to correlate favorably with human judgment \autocite{zheng2023judging}. This methodology is termed the \textbf{LLM-as-a-Judge} method. Frameworks such as \textbf{DeepEval} have materialized to standardize this process. An evaluator LLM is provided with a set of criteria and tasked with assessing the output of the RAG system across various dimensions.

\section{Evaluation Metrics}
When employing an LLM-as-a-Judge, several pivotal metrics are utilized to evaluate the quality of the RAG output. These metrics offer a multi-faceted perspective on performance:
\begin{itemize}
    \item \textbf{G-Eval:} A framework that utilizes LLMs with Chain-of-Thought (CoT) and a form-filling paradigm to assess the quality of NLG outputs \autocite{liu2023geval}. This approach has shown high correlation with human judgments, outperforming conventional reference-based metrics.
    \item \textbf{Directed Acyclic Graph (DAG) metrics:} Within DeepEval, this allows users to create decision trees powered by LLMs for evaluation. Each step in their evaluation process is a node, where an LLM makes a decision or extracts key attributes, and the result determines the subsequent evaluation path. If a response passed one check, it moves on to the next; if it fails, the scoring logic adjusted accordingly. This provides a composite score wherein the evaluator LLM assesses the response based on its level of detail, factual accuracy, and the extent to which it is grounded in the provided context.
    \item \textbf{Answer Relevancy:} Quantifies the degree to which the generated answer addresses the user's original query. A factually correct answer lacks utility if it fails to address the posed question \autocite{deepeval2023}.
    \item \textbf{Faithfulness:} This metric evaluates whether the LLM's response constitutes a faithful representation of the information contained within the retrieved context. A high faithfulness score indicates that the model did not fabricate information not present in the source documents \autocite{deepeval2023}.
    \item \textbf{Hallucination:} Specifically quantifies the incidence of factually incorrect or nonsensical statements within the output. This constitutes a critical metric for constructing trustworthy RAG systems \autocite{deepeval2023}.
\end{itemize}

By systematically testing different prompt structures and styles and evaluating them against these metrics, it becomes feasible to identify the optimal prompt design that maximizes the performance of the RAG system for a given application.